% context and motivation

Energy system models have become a pivotal instrument for policy-making to find
cost-efficient system layouts that satisfy ambitious climate change mitigation
targets. But even though they have proliferated in spatial, temporal,
technological and sectoral detail and scope in recent years, least-cost
optimisation models can easily give a false sense of exactness
\cite{Trutnevyte2016, pye_modelling_2020}. Frequently, they present just a
single least-cost solution for a single set of cost assumptions, which not only
neglects uncertainties inherent to technology cost projections which capacity
expansion models are susceptible to \cite{trondle_trade-offs_2020, yue_review_2018,
pye_assessing_2018}, but also hides a wide array of alternative solutions that
are equally feasible and only marginally more expensive \cite{nearoptimal,
lombardi_policy_2020, sasse_regional_2020}.

% relevance of trade-offs

Trade-offs revealed by deviating from least-cost solutions are extremely
attractive for policymakers, because they allow them to make decisions based on
non-economic criteria without affecting the cost-effectiveness of the system.
Knowing that many similarly costly but technologically diverse solutions exist,
helps to accommodate political and social dimensions that are otherwise hard to
quantify; for instance, rising public opposition towards reinforced transmission
lines and onshore wind turbines or an uneven distribution of new infrastructure
\cite{mccollum_energy_2020,sasse_regional_2020,schlachtberger_cost_2018}.

% how to find trade-offs

Techniques like multi-objective optimisation and
modelling-to-generate-alternatives are designed to find such near-optimal
solutions. Among others, they have been applied to investment planning models of
the European \cite{nearoptimal}, the Italian 
\cite{lombardi_policy_2020}, and the United States power system
\cite{DeCarolis2016}, pathways to decarbonise the power system of the United
Kingdom \cite{Li2017}, a single-node sector-coupling model of Germany
\cite{nacken_integrated_2019}, global integrated assessment models
\cite{Price2017}, and were combined with a quick hull algorithm to span a
polytope of low-cost solutions for a single set of cost parameters
\cite{pedersen_modeling_2020}.

% trade-offs must be robust to cost uncertainty

However, most of the studies above only use a central cost projection for each
considered technology. But recent decades have shown that many of these
projections contain a high level of uncertainty, particularly for fast-moving
technologies like solar, batteries and hydrogen storage. This uncertainty
propagates through the model to strongly affect the optimal and near-optimal
system compositions, thus undermining any analysis of the trade-offs. Hence, it
is crucial that apparent compromises are rigorously tested for robustness to
technology cost uncertainty to raise confidence in conclusions about viable,
cost-effective power system designs. To thoroughly sweep the uncertainty space,
we can fortunately avail of previous works on multi-dimensional global
sensitivity analysis techniques in the context of least-cost optimisation
\cite{trondle_trade-offs_2020, mavromatidis_uncertainty_2018,
pizarro-alonso_uncertainties_2019, fais_impact_2016, usher_value_2015}. We
expand their application to strengthening insights on the scope of near-optimal
trade-offs, wherein the novelty of this contribution lies.

% this work / novelty

Here, we systematically explore robust trade-offs near the cost-optimum of a
fully renewable European electricity system model, and investigate how they are
affected by uncertain technology cost projections. Thereby, we evaluate both
compromises between system cost and technology choices, as well as between pairs
of technologies. We do so by solving numerous spatially and temporally explicit
long-term investment planning problems that coordinate generation, transmission
and storage investments subject to multi-period linear optimal power flow
constraints, while employing methods from global sensitivity analysis to account
for a wide range of cost projections for wind, solar, battery and hydrogen
storage technologies.

% computational achievement

To handle the immense computational burden incurred by searching for
near-optimal alternatives alongside evaluating many different cost parameter
sets, we employ multi-fidelity surrogate modelling techniques, based on sparse
polynomial chaos expansion that allow us to merge results from one simpler and
another more detailed model. This approach has been proven very effective in
Tröndle et al.~\cite{trondle_trade-offs_2020}. Heavy parallelisation with
high-performance computing infrastructure allowed us to solve more than 50,000
resource-intensive optimisation problems which, in combination with surrogate
modelling, admit spanning a probabilistic space of near-optimal solutions rather
than putting single scenarios into the foreground.

% results

Thereby, we are able to present alternative solutions beyond least-cost that
have a high chance of involving a limited cost increase, just as we identify
regions that are unlikely to be cost-efficient. We derive both ranges of options
and technology-specific boundary conditions, that are not affected by cost
uncertainty and must be met to keep the total system cost within a specified
range. Our results show that indeed many such similarly costly but
technologically diverse solutions exist regardless of how technology cost
developments will unfold within the considered ranges.  

% --------------------- unused --------------------------

% - technology cost, useable potentials, discount rate, weather year, climate change, projection and regional distribution electricity demand to only name a few
% These include
% which parameters are uncertain and to what extent,
% what parameters influence outputs the most,
% how uncertainty propagates to the outputs, and
% how to hedge against uncertainty with stochastic or robust optimisation techniques.
% The Karush-Kuhn-Tucker multipliers
% can tell us about the local sensitivity to small changes in the constraints at the optimal point.

% Two ways:
% - expand distribution view of cost-optimal capacities by the space of near-optimal solutions
% - or vice versa: augment the near-optimal solution space by adding variance induced by cost uncertainty
% For instance, they translate a carbon cap to a carbon price, map a renewable generation target to a required subsidy, limited potentials to a scarcity cost, or even output nodal prices \cite{marketvalue}.

% - related: least-cost solutions for one parameter sample may be within 5\% of the optimum for another parameter sample, or vice versa

% Additionally acknowledging sensitivities towards cost assumptions further raises
% confidence in modelling results. 

% While in scenario analysis only
% a handful of potentially biased narratives are developed \cite{DeCarolis2017,soroudi_decision_2013,usher_value_2015},

% If we can understand the key drivers of the
% cost structure and technology preferences
% \cite{usher_value_2015,moret_characterization_2017}, central parameters can be
% targeted for technological learning with policy.

% Manifold techniques exist to assess such parametric uncertainty
% from a variety of different angles.

% Combining the
% effects of uncertainty and near-optimality means that results on technology
% mixes are volatile to small perturbations in the model inputs.

% The structural component to uncertainty can relate to modelling errors
% caused by the linear approximation of power flows \cite{flowandlosses}, spatial aggregation
% \cite{hoersch_spatial_2017}, temporal aggregation \cite{kotzur_tsa_2018},
% technology simplification \cite{DeCarolis2017}, or--as will be studied in this
% contribution--degrees of freedom near the cost-optimum \cite{nearoptimal}.

% Foremost, previous works have focussed
% on either investigating near-optimal solutions or parametric uncertainty.

% However, given that only a small fraction of the uncertainty space
% is sampled by single-parameter variation alone,
% some decisive interactions may be overlooked \cite{pizarro-alonso_uncertainties_2019}.

% in hindsight
% developments in the electricity sector have rarely followed cost-optimal paths
%  \cite{}

% arising

% transmission
% expansion limits \cite{schlachtberger_benefits_2017}, reference weather years
% \cite{bloomfield_2021}, distributional equity requirements
% \cite{sasse_regional_2020,sasse_distributional_2019}, or increasing emission
% reduction targets

% It is the combination of rigorous uncertainty quantification in the space of
% near-optimal solutions that makes the novelty of this contribution.

% Herein lies the novelty of this contribution.

% pan-continental