%\blindtext

% perspectives papers

energy system models as a tool for policy-making

find cost-efficient system layouts that satisfy climate protection ambitions

"Energy modellers should explore extremes more systematically in scenarios" e.g. rising social resistance to certain technologies, suddenly falling technology cotsts \cite{mccollum_energy_2020,pye_modelling_2020}

do more uncertainty analysis \cite{yue_review_2018, pye_assessing_2018, Pfenninger2014,pye_modelling_2020,decarolis_leveraging_nodate} 

do more sensitivity analysis to increase confidence \cite{bistline_deepening_2020} 

% motivation / challenges
Flat directions near the optimum are very important for policymakers,
because they allow them to make decisions based on other,
non-economic criteria, without affecting the cost-effectiveness of the system.

social opposition to onshore wind or transmisison expansion

Few parameters are really relevant, what determines the cost structure \cite{moret_characterization_2017} \cite{usher_value_2015}

Focus on cost inputs and constraints affecting outputs relevant to public acceptance.

Explorative analysis: explore more, communicate better, what if questions, find key drivers

derive robust policy guidance, how to handle uncertainties 

Where in parameter space is it cheap? What's the danger zone?
- Or are 90\% of the uncertainty space within 10\% of the least cost solutions?
- decision making under deep uncertainty:

Models can:
- easily give a false sense of exactness, not only because they neglect uncertainties, non-linearities, non-convexities
- underestimate social factors (behaviour, concerns about nuclear, transmission, wind)

% uncertainty

Different time domains of uncertainty:
- short-term: operation, e.g. weather forecasts
- long-term: planning, e.g. technology cost

Difference between parametric and structural uncertainty (extended from \cite{DeCarolis2017})
- parametric: technology cost, useable potentials, discount rate, weather year, climate change, electricity demand
- structural: power flow approximation, spatial and temporal aggregation, technology simplification (e.g. UC)

% uncertainty vs sensitivity

Difference between uncertainty and sensitivity analsis following \cite{usher_value_2016}:
- uncertainty characterization/analysis: which parameters are uncertain and to what extent
- uncertainty propagation: to what extent does uncertainty exist in the outputs
- sensitivity analysis: what input parameters influence the outputs the most
- stochastic and robust optimisation: how to hedge against uncertainty, embedded in optimisation
%\url{https://uncertainpy.readthedocs.io/en/latest/theory.html}

% review of uncertainty analysis methods

literature sort into below categories
\cite{fais_impact_2016} GSA path dependency, critical technologies, complementarities + substitutes
\cite{mavromatidis_uncertainty_2018} GSA
\cite{pilpola_analyzing_2020}
\cite{schyska_sensitivity_2020}
\cite{soroudi_decision_2013}
\cite{lopion_cost_2019}
\cite{moret_robust_2016}

point-estimate method \cite{soroudi_decision_2013}

Karush-Kuhn-Tucker multipliers
- In optimisation theory the KKT equations can tell us about the sensitivity of the optimal point to small changes in the constraints.
- Examples: carbon cap + carbon price, renewable generation target + required subsidy, limited potentials (onshore wind, transmission) + scarcity cost, locational marginal prices
- but no info on global behaviour of the objective function on the feasible space, which can be highly non-linear.

scenario-based sensitivity analysis
- storyline with narrative elements \cite{DeCarolis2017}
- criticised as crude, local, subjective \cite{usher_value_2015}
- but more qualified and contextual description possible

local sensitivity analysis
- parameter sweeps one dimension at a time \cite{schlachtberger_cost_2018}
- in the space of costs and constraint constants
- main interdependencies may already be covered \cite{schlachtberger_cost_2018}
- but only small fraction of uncertainty space is sampled
- may overlook influential parameters \cite{pizarro-alonso_uncertainties_2019}
- examples: spatial resolution \cite{hoersch_spatial_2017}, temporal resolution \cite{kotzur_tsa_2018}, investement cost \cite{shirizadeh_how_2019}, potentials (offwind, onwind, solar) \cite{schlachtberger_cost_2018}, line expansion limits \cite{schlachtberger_benefits_2017}, weather years \cite{bloomfield_2021}, equity requirements \cite{sasse_regional_2020,sasse_distributional_2019}, emission reduction targets

global sensitivity analysis
- changing multiple paramters at a time; co-varying inputs; multi-parameter variations
- to identify interesting scenarios \cite{usher_value_2015}
- "quantifying the effects of random input variables onto the variance of the response of a mathematical model" \cite{sudret_global_2008}
- elementary effects method (Morris screening, one-at-a-time) to counteract high computational requirements \cite{usher_value_2015} \cite{pizarro-alonso_uncertainties_2019} \cite{moret_robust_2016}
- surrogate models using polynomial chaos expansion
- review \cite{iooss_review_2014}

Multi-Objective Optimisation
%- \url{https://en.wikipedia.org/wiki/Multi-objective_optimization}
%- \url{https://en.wikipedia.org/wiki/Pareto_efficiency}
- \cite{nearoptimal}
- epsilon constraint method
- determine Pareto-fronts from multi-objective optimisation \cite{mavrotas_effective_2009}
- fuzzy c-shaped plots are empirical attainment function \cite{binois_quantifying_2015} \cite{de_cursi_uncertainty_2021}

Modelling-to-Generate Alternatives
\cite{DeCarolis2016}
\cite{Price2017}
\cite{nacken_integrated_2019} sector coupling
\cite{pedersen_modeling_2020} density, 2D span of near-optimal feasible space

% relation between structural and parametric uncertainty

marrying near-optimal analysis with parametric uncertainty
- least-cost solutions for one parameter sample may be within 5\% of the optimum for another parameter sample, or vice versa
- "resilient SPORES are relatively insensitive to technology cost" \cite{lombardi_policy_2020}
- EXPANSE combines MGA-type and MC-type uncertainty analysis in total 800 solutions \cite{Li2017} \cite{Trutnevyte2013}

% this work / novelty

European scope

fully renewable power system

long-term investment planning

coordinating generation, transmission and storage

enhances of \cite{nearoptimal}

while we look at aggregate system capacities of certain technologies, the individual investment decisions composing this sum are spatially explicit!

Cost uncertainty can be influenced:
- technological learning and rate of deployment are unknown
- would be useful to identify regions in the parameter space which are very cheap
- can aim for them with policy; "controllable uncertainty"

we find solutions which have a very high chance of being within X\% of the cost-optimum

conversely we also identify regions which are unlikely to be cost-efficient

represent probability with which capacitiies of system components are contained
within the near-optimal space

<range of options> and technology-specific <boundary conditions> robust to cost uncertainties (individually and in pairs)

persistance!

Span a space of options for society and politics to work with

as a bridge between optimisation modelling and decision making

incorporating unknowns and options arising by deviating from least-cost
boosts these models policy relevance

%one of the best ways to make system cheaper is just to make offshore wind cheaper, which has high acceptance and big cost reduction potentials AND it has one of the biggest impacts on TSC.