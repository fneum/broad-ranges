% energy system modelling challenges and robustness

energy system models as a tool/instrument for policy-making
- find cost-efficient system layouts that satisfy climate protection ambitions
- models can easily give a false sense of exactness, not only because they neglect uncertainties and simplify
- models can underestimate social factors (behaviour, concerns about / opposition to nuclear, transmission, wind)

presenting just a single least-cost solution neglects an large degree of freedom \cite{nearoptimal}

Alternative options arising by deviating from least-cost solution are very important for policymakers,
because they allow them to make decisions based on other, non-economic criteria,
without affecting the cost-effectiveness of the system.

- acknowledge uncertainties raises confidence in modelling results and able to derive robust policy guidance
- understand key drivers of the cost structure and technology focus \cite{usher_value_2015,moret_characterization_2017}
- help to accomodate political and social dimensions that are otherwise hard to quantify
- span a space of options for society and politics to work with beyond least-cost
- enable social trade-offs, build a consensus

- identify cheap regions in parameter space, can aim for them with policy to force technological learning
- act as a bridge between optimisation modelling and decision making
- knowing that near the optimum many similarly costly but technologically diverse solutions exist leaves room for political discussion and compromises

% uncertainty in general

Calls for more
- exploration of extremes more systematically e.g. rising social resistance to certain technologies, suddenly falling technology cotsts \cite{mccollum_energy_2020,pye_modelling_2020}
- uncertainty analysis \cite{Pfenninger2014,yue_review_2018,pye_assessing_2018,pye_modelling_2020,decarolis_leveraging_nodate} 
- sensitivity analysis \cite{bistline_deepening_2020,trondle_trade-offs_2020} 

Difference between uncertainty and sensitivity analsis \cite{usher_value_2016}:
- uncertainty characterization/analysis: which parameters are uncertain and to what extent
- uncertainty propagation: to what extent does uncertainty exist in the outputs
- sensitivity analysis: what input parameters influence the outputs the most
- stochastic and robust optimisation: how to hedge against uncertainty, embedded in optimisation

% parametric uncertainty

Examples:
- technology cost, useable potentials, discount rate, weather year, climate change, electricity demand
- different time domains of uncertainty: short-term: operation, e.g. weather forecasts; long-term: planning, e.g. technology cost

Karush-Kuhn-Tucker multipliers
- can reveal the sensitivity to small changes in the constraints at the optimal point.
- Examples: translate carbon cap + carbon price, renewable generation target + required subsidy, limited potentials (onshore wind, transmission) + scarcity cost, locational marginal prices
- but no info on global behaviour of the objective function on the feasible space, which can be highly non-linear.
    
scenario-based sensitivity analysis
- storyline with narrative elements \cite{DeCarolis2017}
- but more qualified and contextual description possible \cite{soroudi_decision_2013}
- criticised as biased, unstructured, subjective \cite{usher_value_2015}

local sensitivity analysis
- parameter sweeps one dimension at a time \cite{schlachtberger_cost_2018}
- across costs and constraint constants
- main interdependencies may already be covered \cite{schlachtberger_cost_2018}
- but given that only small fraction of uncertainty space is sampled it may overlook effects \cite{pizarro-alonso_uncertainties_2019}
- very popular: examples: spatial resolution \cite{hoersch_spatial_2017}, temporal resolution \cite{kotzur_tsa_2018}, investement cost \cite{shirizadeh_how_2019}, potentials (offwind, onwind, solar) \cite{schlachtberger_cost_2018}, line expansion limits \cite{schlachtberger_benefits_2017}, weather years \cite{bloomfield_2021}, equity requirements \cite{sasse_regional_2020,sasse_distributional_2019}, emission reduction targets
- \cite{schyska_sensitivity_2020}

global sensitivity analysis
- changing multiple paramters at a time; co-varying inputs; multi-parameter variations
- review \cite{iooss_review_2014}
- to identify interesting scenarios \cite{usher_value_2015}
- determine how random input variables affect variance of outputs \cite{sudret_global_2008}
- methods to counteract high computational requirements \cite{usher_value_2015} \cite{pizarro-alonso_uncertainties_2019} \cite{moret_robust_2016}
- surrogate models using polynomial chaos expansion \cite{trondle_trade-offs_2020}
- applications: \cite{fais_impact_2016,mavromatidis_uncertainty_2018,pilpola_analyzing_2020}

% structural uncertainty

Examples \cite{DeCarolis2017}:
- power flow approximation, spatial and temporal aggregation, technology simplification (e.g. UC), unmodelled objectives

Multi-Objective Optimisation
- epsilon constraint method used in \cite{nearoptimal}
- determine Pareto-fronts from multi-objective optimisation \cite{mavrotas_effective_2009}

Modelling-to-Generate Alternatives
- \cite{DeCarolis2016} US electricity sector
- \cite{Price2017} integrated assessment model
- \cite{nacken_integrated_2019} single node sector coupling
- \cite{pedersen_modeling_2020} density, 2D span of low-cost space

%structural AND parametric uncertainty

marrying near-optimal analysis with parametric uncertainty
- both have an effect on confidence in conclusions
- related: least-cost solutions for one parameter sample may be within 5\% of the optimum for another parameter sample, or vice versa
- only few: MGA including cost uncertainty \cite{Trutnevyte2013,Li2017,lombardi_policy_2020}

% this work / novelty

work with a set of more vague but also more robust, viable alternatives
- systematically explore decision and parameter space
- range of options and technology-specific boundary conditions robust to cost uncertainties
- we find solutions which have a very high chance of being within X\% of the cost-optimum
- conversely we also identify regions which are unlikely to be cost-efficient
- represent probability with which capacities of system components are contained within the low-cost space
- expand distribution view of cost-optimal capacities by the space of near-optimal solutions
- or vice versa: augment the near-optimal solution space by adding variance induced by cost uncertainty

model characteristics
- European scope
- fully renewable power system
- long-term investment planning
- coordinating generation, transmission and storage investments
- spatially and temporally explicit
- combine low/high fidelity models and surrogate models

impact