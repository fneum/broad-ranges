% energy system modelling challenges

Energy system models have become a pivotal instrument for policy-making
to find cost-efficient system layouts that satisfy ambitious climate change mitigation targets.
But even though they have proliferated in spatial, temporal, technological and sectoral detail and scope in recent years,
least-cost optimisation models can easily give a false sense of exactness.
Prominently, they may underplay social factors, such as
rising public opposition to transmission lines, wind, nuclear, or biomass
or an uneven distribution of infrastructure \cite{mccollum_energy_2020,sasse_regional_2020}.

Presenting just a single least-cost solution for a single set of cost assumptions
not only neglects uncertainties inherent to technology cost projections the
investment planning models are very sensitive to
\cite{trondle_trade-offs_2020,Pfenninger2014,yue_review_2018,pye_assessing_2018,pye_modelling_2020},
but also hides a wide array of alternative solutions
that are equally feasible and only marginally more expensive
\cite{nearoptimal,lombardi_policy_2020,sasse_distributional_2019}.

% why are alternative solutions important

Such alternatives arising by deviating from least-cost solutions are attractive for policymakers,
because they allow them to make decisions based on other, non-economic criteria,
without affecting the cost-effectiveness of the system.
In other words, knowing that many similarly costly but technologically diverse solutions
exist helps to accommodate political and social dimensions that are otherwise hard to quantify.
Additionally acknowledging sensitivities towards cost assumptions further raises confidence
in modelling results. If we can understand the key drivers of the cost structure and
technology preferences \cite{usher_value_2015,moret_characterization_2017},
central parameters can be targeted for technological learning with policy.

% parametric uncertainty

% - technology cost, useable potentials, discount rate, weather year, climate change, projection and regional distribution electricity demand to only name a few

Manifold techniques exist to assess such parametric uncertainty
from a variety of different angles.
% These include
% which parameters are uncertain and to what extent,
% what parameters influence outputs the most,
% how uncertainty propagates to the outputs, and
% how to hedge against uncertainty with stochastic or robust optimisation techniques.
The Karush-Kuhn-Tucker multipliers
can tell us about the local sensitivity to small changes in the constraints at the optimal point.
For instance, they translate a carbon cap to a carbon price, map a renewable generation target to a required subsidy, limited potentials to a scarcity cost, or even output nodal prices \cite{marketvalue}.
While in scenario analysis only
a handful of potentially biased narratives are developed \cite{DeCarolis2017,soroudi_decision_2013,usher_value_2015}, local sensitivity analysis sweeps the parameter space
across costs and constraint constants one dimension at a time \cite{schlachtberger_cost_2018}.
Popular target parameters include the investement cost and potentials \cite{schlachtberger_cost_2018},
transmission expansion limits \cite{schlachtberger_benefits_2017},
reference weather years \cite{bloomfield_2021},
distributional equity requirements \cite{sasse_regional_2020,sasse_distributional_2019},
or varying emission reduction targets.
But given that only a small fraction of the uncertainty space
is sampled by single-parameter variation alone,
decisive interactions may be overlooked \cite{pizarro-alonso_uncertainties_2019}.
Global sensitivity analysis techniques, in turn, vary multiple parameters simultaneously
and analyse how each random input variable affects the variance of outputs \cite{sudret_global_2008},
but easily implicate high computational requirements due to the required sampling size \cite{usher_value_2015,pizarro-alonso_uncertainties_2019,moret_robust_2016}.
Prominent applications in the context of energy system optimisation include among others
\cite{trondle_trade-offs_2020,fais_impact_2016,mavromatidis_uncertainty_2018,pilpola_analyzing_2020}.

% structural uncertainty (+ uncertainty)
\newpage
Besides parametric uncertainty,
there also exists a structural component to uncertainty, that can
relate to modelling errors caused by
linear approximation of powerflow \cite{flowandlosses},
spatial aggregation \cite{hoersch_spatial_2017},
temporal aggregation \cite{kotzur_tsa_2018},
technology simplification \cite{DeCarolis2017},
or--as will be studied in this contribution--degrees
of freedom near the cost-optimum \cite{nearoptimal}.
Techniques like multi-objective optimisation and modelling-to-generate-alternatives
are designed to find such near-optimal alternatives and have been
applied to integrated assessment models \cite{Price2017}, 
the United States electricity sector \cite{DeCarolis2016},
the Italian power system \cite{lombardi_policy_2020},
pathways to decarbonise the power system of the United Kingdom \cite{Li2017},
a single node sector-coupling model of Germany \cite{nacken_integrated_2019},
and were combined with a quick hull algorithm to span a polytope of low-cost solutions
for a single set of cost parameters \cite{pedersen_modeling_2020}.
Previous works have mostly % national models only
focussed on either investigating near-optimal solutions or parametric uncertainty
(excluding \cite{Li2017,lombardi_policy_2020}).
But both alike influence the confidence in conclusions
about viable future power system designs and it is crucial
to reflect on such compromises and trade-offs that exist despite prevailing cost uncertainty.
% - related: least-cost solutions for one parameter sample may be within 5\% of the optimum for another parameter sample, or vice versa

% this work / novelty

In this contribution, we systematically explore the space of near-optimal solutions
of a fully renewable pan-continental European electricity system model
and investigate how it is affected by uncertain technology cost projections.
That means we solve spatially and temporally explicit long-term investment planning problems that
coordinate generation, transmission and storage investments
subject to multi-period linear optimal power flow constraints,
while accounting for a wide range of cost projections for wind, solar, battery and
hydrogen storage technologies.
To handle the computational burden incurred by searching for near-optimal solutions
alongside evaluating many different cost parameter sets,
we employ multi-fidelity surrogate modelling techniques based on sparse polynomial chaos expansion
that allow us to merge results from one simpler and another more detailed model.
As we solve more than 50,000 resource-intensive optimisation problems,
our analysis has only become possible in recent years by the cheap availability 
of high-performance computing resources. 

% Two ways:
% - expand distribution view of cost-optimal capacities by the space of near-optimal solutions
% - or vice versa: augment the near-optimal solution space by adding variance induced by cost uncertainty

% results

We investigate alternative solutions beyond least-cost
that have a high chance of involving a limited cost increase,
just as we identify regions that are unlikely to be cost-efficient.
We derive both, ranges of options and technology-specific boundary conditions,
that are not affected by cost uncertainty and must be 
met to keep the total system cost within a specified range.
Our results show that many such similarly costly,
but technologically diverse solutions exist regardless
how technology cost developments will unfold.

