% energy system modelling challenges

energy system models 
- a tool/instrument for policy-making
- find cost-efficient system layouts that satisfy ambitions for climate change mitigation
- proliferated in spatial, temporal and sectoral detail
- least-cost optimisation models can easily give a false sense of exactness.
- caveats simplify, neglect uncertainties, underestimate social factors
- rising social resistance / concerns about / public opposition to certain technologies such as transmission, wind, nuclear, biomass \cite{mccollum_energy_2020} or uneven distribution of infrastructure

presenting just a single least-cost solution
- neglects a wide variety of alternative solutions that are equally feasible and only marginally more expensive \cite{nearoptimal,lombardi_policy_2020,sasse_distributional_2019,trondle_trade-offs_2020}
- underplays uncertainties regarding cost projections, investment planning models are are known to be very sensitive to \cite{Pfenninger2014,yue_review_2018,pye_assessing_2018,pye_modelling_2020} 

% why are alternative solutions important

Alternatives arising by deviating from least-cost solution are very important for policymakers,
because they allow them to make decisions based on other, non-economic criteria,
without affecting the cost-effectiveness of the system.

acknowledge uncertainties raises confidence in modelling results and able to derive robust policy guidance
- knowing that many similarly costly but technologically diverse solutions exist helps to accomodate political and social dimensions that are otherwise hard to quantify
- understand key drivers of the cost structure and technology focus \cite{usher_value_2015,moret_characterization_2017}
- identify cheap regions in parameter space, can aim for them with policy to induce technological learning

% parametric uncertainty

Examples:
- technology cost, useable potentials, discount rate, weather year, climate change, electricity demand
- different time domains of uncertainty: short-term: operation, e.g. weather forecasts; long-term: planning, e.g. technology cost

different aspects \cite{usher_value_2016}:
- which parameters are uncertain and to what extent
- what parameters influence outputs the most
- how uncertainty propagates to the outputs outputs
- how to hedge against uncertainty with stochastic or robust optimisation

Karush-Kuhn-Tucker multipliers
- can reveal the sensitivity to small changes in the constraints at the optimal point.
- but not the global behaviour on the feasible space, which can be highly non-linear.
- Examples: translate carbon cap + carbon price, renewable generation target + required subsidy, limited potentials (onshore wind, transmission) + scarcity cost, locational marginal prices
    
scenario-based sensitivity analysis
- handful of storylines with descriptive narrative elements \cite{DeCarolis2017,soroudi_decision_2013}
- criticised as biased and unstructured \cite{usher_value_2015}

local sensitivity analysis
- parameter sweeps across costs and constraint constants changing one dimension at a time \cite{schlachtberger_cost_2018}
- premise that main interdependencies may already be covered \cite{schlachtberger_cost_2018}
- very popular: investement cost and potentials (offwind, onwind, solar) \cite{schlachtberger_cost_2018}, line expansion limits \cite{schlachtberger_benefits_2017}, reference weather years \cite{bloomfield_2021}, distributional equity requirements \cite{sasse_regional_2020,sasse_distributional_2019}, emission reduction targets
- but given that only small fraction of uncertainty space is sampled it may overlook effects \cite{pizarro-alonso_uncertainties_2019}

global sensitivity analysis
- vary multiple paramters at a time
- determine how each random input variable affects the variance of outputs \cite{sudret_global_2008}
- high computational requirements \cite{usher_value_2015,pizarro-alonso_uncertainties_2019,moret_robust_2016}
- applications: \cite{trondle_trade-offs_2020,fais_impact_2016,mavromatidis_uncertainty_2018,pilpola_analyzing_2020}

% structural uncertainty

Examples \cite{DeCarolis2017}:
- power flow approximation \cite{flowandlosses}, spatial aggregation \cite{hoersch_spatial_2017}, temporal aggregation \cite{kotzur_tsa_2018}, technology simplification (e.g. UC), unmodelled objectives

Multi-Objective Optimisation and Modelling-to-Generate Alternatives
- epsilon constraint method used in \cite{nearoptimal}
- determine Pareto-fronts \cite{mavrotas_effective_2009}
- find near-optimal solutions
- \cite{Price2017} integrated assessment model
- \cite{DeCarolis2016} US electricity sector
- \cite{nacken_integrated_2019} single node sector coupling Germany
- \cite{pedersen_modeling_2020} combination of MGA with quickhull algorithm to span a polytope of low-cost solutions

%structural AND parametric uncertainty

marrying near-optimal analysis with parametric uncertainty
- both have an effect on confidence in conclusions
- related: least-cost solutions for one parameter sample may be within 5\% of the optimum for another parameter sample, or vice versa
- only few: MGA including cost uncertainty \cite{Trutnevyte2013,Li2017,lombardi_policy_2020}

% this work / novelty

here we
- provide a methodology
- provide (exemplary) results that can be drawn

work with a set of more vague but also more robust, viable alternatives
- systematically explore decision and parameter space
- range of options and technology-specific boundary conditions robust to cost uncertainties
- we find solutions which have a very high chance of being within X\% of the cost-optimum
- conversely we also identify regions which are unlikely to be cost-efficient
- represent probability with which capacities of system components are contained within the low-cost space
- expand distribution view of cost-optimal capacities by the space of near-optimal solutions
- or vice versa: augment the near-optimal solution space by adding variance induced by cost uncertainty
- act as a bridge between optimisation modelling and decision making

show this with model characteristics
- European scope
- fully renewable power system
- based on wind and solar
- long-term investment planning
- coordinating generation, transmission and storage investments
- spatially and temporally explicit
- combine low/high fidelity models and surrogate models
- open-source

impact
- why now? computational power is available
- common areas are robust and large
- large degree of freedom regardless of cost uncertainty