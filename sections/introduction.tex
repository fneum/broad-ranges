% energy system modelling challenges

Energy system models have become a pivotal instrument for policy-making
to find cost-efficient system layouts that satisfy ambitious targets for climate change mitigation.
But even though they have proliferated in spatial, temporal, technological and sectoral detail and scope in recent years,
least-cost optimisation models can easily give a false sense of exactness.
For example, they involve the danger of underplaying social factors
not contained by the model formulation, such as
rising public opposition to certain technologies like transmission, wind, nuclear, or biomass
or an uneven distribution of infrastructure \cite{mccollum_energy_2020,sasse_regional_2020}.

Presenting just a single least-cost solution for a single set of cost assumptions
not only underplays uncertainties inherent to technology cost projections the
investment planning models are are known to be very sensitive to
\cite{trondle_trade-offs_2020,Pfenninger2014,yue_review_2018,pye_assessing_2018,pye_modelling_2020},
but also neglects a wide variety of alternative solutions
that are equally feasible and only marginally more expensive
\cite{nearoptimal,lombardi_policy_2020,sasse_distributional_2019}.

% why are alternative solutions important

Such alternatives arising by deviating from the least-cost solution are attractive for policymakers,
because they allow them to make decisions based on other, non-economic criteria,
without affecting the cost-effectiveness of the system.
In other words, knowing that many similarly costly but technologically diverse solutions
exist helps to accomodate political and social dimensions that are otherwise hard to quantify.
Additionally acknowledging sensitivities towards cost assumptions further raises confidence
in modelling results. If we can understand the key drivers of the cost structure and
technology preferences \cite{usher_value_2015,moret_characterization_2017},
central parameters can be targeted for technological learning with policy.

% parametric uncertainty

% - technology cost, useable potentials, discount rate, weather year, climate change, projection and regional distribution electricity demand to only name a few

Numerous examples of parametric uncertainty analysis:
assessed from different angles:
- which parameters are uncertain and to what extent
- what parameters influence outputs the most
- how uncertainty propagates to the outputs outputs
- how to hedge against uncertainty with stochastic or robust optimisation
Karush-Kuhn-Tucker multipliers
can tell us about the local sensitivity to small changes in the constraints at the optimal point,
but not the global behaviour on the feasible space, which may be highly non-linear.
For instance, these dual variables translate a carbon cap to a carbon price, map a renewable generation target to a required subsidy, limited potentials to a scarcity cost, or even output nodal prices \cite{marketvalue}.
While in scenario-based sensitivity analysis only
a handful of potentially biased narratives are developed \cite{DeCarolis2017,soroudi_decision_2013,usher_value_2015}, local sensitivity analysis sweeps the parameter space
across costs and constraint constants altering only one dimension at a time \cite{schlachtberger_cost_2018}.
Popular target parameters include the investement cost and potentials \cite{schlachtberger_cost_2018},
transmission expansion limits \cite{schlachtberger_benefits_2017},
reference weather years and climate change \cite{bloomfield_2021,impactofclimatechange},
distributional equity requirements \cite{sasse_regional_2020,sasse_distributional_2019},
or varying emission reduction targets.
But given that only a small fraction of the uncertainty space
is sampled by single-parameter variation alone,
decisive interactions may be overlooked \cite{pizarro-alonso_uncertainties_2019}.

global sensitivity analysis
- vary multiple paramters at a time
- analyses how each random input variable affects the variance of outputs \cite{sudret_global_2008}
- high computational requirements \cite{usher_value_2015,pizarro-alonso_uncertainties_2019,moret_robust_2016}
- applications: \cite{trondle_trade-offs_2020,fais_impact_2016,mavromatidis_uncertainty_2018,pilpola_analyzing_2020}
although account for uncertainty, not compromises and trade-offs

% structural uncertainty

Besides parametric uncertainty, there is also a component structural uncertainties of the
optimisation model.
Examples :
- power flow approximation \cite{flowandlosses}, spatial aggregation \cite{hoersch_spatial_2017}, temporal aggregation \cite{kotzur_tsa_2018}, technology simplification \cite{DeCarolis2017}, or--as will be studied in this contribution--degrees of freedom near the cost-optimum \cite{nearoptimal}.
Techniques like multi-objective optimisation and modelling-to-generate-alternatives
- epsilon constraint method used in \cite{nearoptimal}
- determine Pareto-fronts \cite{mavrotas_effective_2009}
- find near-optimal solutions
- applied in for integrated assessment models \cite{Price2017}, 
for the United States electricity sector \cite{DeCarolis2016},
the Italian power system \cite{lombardi_policy_2020},
a single node sector-coupling model of Germany \cite{nacken_integrated_2019},
and combined with a quickhull algorithm to span a polytope of low-cost solutions \cite{pedersen_modeling_2020}.

%structural AND parametric uncertainty

Previous works have mostly put focus on either near-optimal or parametric uncertainty
- both have an effect on confidence in conclusions
- related: least-cost solutions for one parameter sample may be within 5\% of the optimum for another parameter sample, or vice versa
- only few: MGA including cost uncertainty \cite{Trutnevyte2013,Li2017,lombardi_policy_2020}

% this work / novelty

In this contribution, we systematically explore parameter space of technology cost projections and
investigate how they affect the space of near-optimal solutions
of a fully renewable European electricity system model
that is both spatially and temporally resolved. % open source
That means we solve long-term investment planning problems that
coordinate generation, transmission and storage investments,
while accounting for a wide range of cost projections for wind, solar, battery and
hydrogen storage technologies.
To handle the computational burden incurred by searching for near-optimal solutions
alongside evaluating many different cost parameter sets,
we employ multi-fidelity surrogate modelling techniques based on sparse polynomial chaos expansion
that allow us to merge results from a simpler and a more detailed model.
As we solve more than 50,000 resource-intensive optimisation problems,
this kind of analysis has only become possible by the low-cost availability 
of high-performance computing resources. 

% Two ways:
% - expand distribution view of cost-optimal capacities by the space of near-optimal solutions
% - or vice versa: augment the near-optimal solution space by adding variance induced by cost uncertainty

% results

We investigate alternative solutions beyond least-cost
that have a high chance of involving a limited cost increase,
just as we identify regions that are unlikely to be cost-efficient.
We derive both, ranges of options and technology-specific boundary conditions,
that are not affected by cost uncertainty and must be 
met to keep the total system cost within a specified range.
Our results show that many such similarly costly,
but technologically diverse solutions exist regardless
how technology cost developments will unfold.

