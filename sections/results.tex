In this section,

\subsection{Least-Cost Solutions}
%%\blindtext

%F violin plots
\begin{SCfigure}
    \includegraphics[width=0.75\textwidth]{violins/violin-capacities-high-prediction.pdf}
    \caption{Lorem ipsum dolor sit amet, consetetur sadipscing elitr, sed diam nonumy eirmod tempor invidunt ut labore et dolore magna aliquyam erat, sed diam voluptua.}
    \label{fig:violin}
\end{SCfigure}

\cref{fig:violin}
propagation of input uncertainties into outputs (when system cost is minimised)
tsc:
- total system cost range between 160 and 220
- means most pessimistic cost is about 40\% more expensive than using optimistic cost
- bell-like probability distribution
generation:
- all solutions build at least 350 GW of solar and 600 GW of wind
- but no more than 1100 GW of wind
- wind tendency towards higher values, solar towards lower values
- least-cost solutions prefer onshore over offshore (most solutions beyond 800 GW onshore, less than 100 GW offshore)
- but alongside battery storage capacity onshore wind capacity has the highest uncertainty range
storage:
- no least-cost solutions without hydrogen, but some without battery
- hydrogen tends to dominate battery storage, except when battery storage is very cheap
- battery also has wider uncertainty range
transmission:
- least affected by cost uncertainties
- almost doubling in relation to today's capacities

criticise that
- interpretation of boundaries is limited
- are not robust when looking beyond least-cost solution and marginal cost penalties are allowed
- does not tell which input uncertainties are responsible (sensitivity analysis)

%F sensitivity of built capacities

\begin{figure}
    \begin{subfigure}[t]{0.32\textwidth}
        \caption{onshore wind}
        \includegraphics[width=\textwidth]{1D/1D-onwind-onwind-high.pdf}
    \end{subfigure}
    \begin{subfigure}[t]{0.32\textwidth}
        \caption{offshore wind}
        \includegraphics[width=\textwidth]{1D/1D-offwind-offwind-high.pdf}
    \end{subfigure}
    \begin{subfigure}[t]{0.32\textwidth}
        \caption{solar}
        \includegraphics[width=\textwidth]{1D/1D-solar-solar-high.pdf}
    \end{subfigure} \\
    \begin{subfigure}[t]{0.32\textwidth}
        \caption{battery storage}
        \includegraphics[width=\textwidth]{1D/1D-battery-battery-high.pdf}
    \end{subfigure}
    \begin{subfigure}[t]{0.32\textwidth}
        \caption{hydrogen storage}
        \includegraphics[width=\textwidth]{1D/1D-H2-H2-high.pdf}
    \end{subfigure}
    \begin{subfigure}[t]{0.32\textwidth}
        \caption{transmission}
        \includegraphics[width=\textwidth]{1D/1D-transmission-H2-high.pdf}
    \end{subfigure}
    \caption{Lorem ipsum dolor sit amet, consetetur sadipscing elitr, sed diam nonumy eirmod tempor invidunt ut labore et dolore magna aliquyam erat, sed diam voluptua.}
    \label{fig:sensitivity}
\end{figure}

\cref{fig:sensitivity}
how:
- like local sensitivity analysis, reduce one dimension from uncertainty space
- sweep the parameter space in one dimension and obserse corresponding capacity
- for instance onshore cost on onshore capacity (presumed highest sensitivity)
- the remaining uncertainty (from other cost assumptions) is depicted by percentiles
- the overall tendency is trivial: the cheaper a technology gets, the more it is built
- but changes of slope and effects on uncertainty range across the parameter space are interesting!
result:
- slope:
  - battery  becomes economically significantly more attractive (see change of slope) once annuity falls below 75 EUR/kW/a
  - [translate into an overnight investment with exemplary lifetime/interest] (regardless of other cost uncertainties), below it is a minor to hydrogen
  - hydrogen on the other hand -> steady slope and uncertainty band
- uncertainty:
  - low cost of onshore wind -> much onshore wind with low uncertainty
  - high cost of onshore wind -> how much is built greatly depends on other cost parameters
  - the opposite for offshore wind (and solar)
- other:
  - the limited uncertainty about optimal levels of network expansion is mostly due to hydrogen storage cost. As cost of storing H2 falls, less grid reinforcment chosen (can also see this  in Sobol indices)
transition:
- these self-sensitivities are only a selection
- we can formalise how input uncertainties affect each outcome with sensitivity indices
- next Sobol

%F sobol indices

\begin{figure}
    \begin{subfigure}[t]{0.45\textwidth}
        \caption{first-order Sobol indices}
        \label{fig:sobol:first}
        \includegraphics[width=\textwidth]{sobol/sobol-m-high.pdf}
    \end{subfigure}
    \begin{subfigure}[t]{0.54\textwidth}
        \caption{total Sobol indices}
        \label{fig:sobol:total}
        \includegraphics[width=\textwidth]{sobol/sobol-t-high-bar.pdf}
    \end{subfigure}
    \caption{Sobol sensitivity indices}
    \label{fig:sobol}
\end{figure}

\cref{fig:sobol} \cref{fig:sobol:first} \cref{fig:sobol:total}
- two ways to display as a heatmap or as stacked bar chart
- first-order effects dominate, but there are second-order effects
- tsc largely decided by how expensive it is to build onshore wind capacity, followed by cost of hydrogen storage
- wind deploymend almost exclusively governed by cost of onwind/offwind
- other carriers yield a more varied picture (depend on other technologies)
- solar deployment also depends on onshore wind and battery cost
- amount of hydrogen influenced by battery and H2 cost alike
- The Sobol indices of the least-cost solution are widely shared across the near-optimal solutions with little deviation.

\subsection{Near-Optimal Solutions}
%%\blindtext

- uncertainty quantification of outputs at least-cost solutions: check
- sensitivity analysis of inputs on outputs at least-cost solutions: check
- has been shown: for single-cost parameter set a wide array of technologically diverse but similarly costly solutions exist
- next we want to examine where agree/disagree, feasible alternatives common to all, few or no parameter sets
- robust conclusions on near-optimal solutions that are not affected by uncertainty: next
- in other words: find solutions that are likely within epsilon of the least-cost solution
- combination of cost uncertainty and slack leads to: look at solutions between 160 (optimistic least-cost) and 240 (pessimistic with maximum slack) bn EUR pa

%F fuzzy c-shaped plots

\begin{figure}
    \noindent\makebox[\textwidth]{
    \begin{subfigure}[t]{0.45\textwidth}
        \centering
        % \caption{any wind}
        \includegraphics[width=\textwidth]{neardensity/surr-high-wind.pdf}
    \end{subfigure}
    \begin{subfigure}[t]{0.45\textwidth}
        \centering
        % \caption{onshore wind}
        \includegraphics[width=\textwidth]{neardensity/surr-high-onwind.pdf}
    \end{subfigure}
    \begin{subfigure}[t]{0.45\textwidth}
        \centering
        % \caption{offshore wind}
        \includegraphics[width=\textwidth]{neardensity/surr-high-offwind.pdf}
    \end{subfigure}
    }
    \noindent\makebox[\textwidth]{
    \begin{subfigure}[t]{0.45\textwidth}
        \centering
        % \caption{solar}
        \includegraphics[width=\textwidth]{neardensity/surr-high-solar.pdf}
    \end{subfigure}
    \begin{subfigure}[t]{0.45\textwidth}
        \centering
        % \caption{transmission network}
        \includegraphics[width=\textwidth]{neardensity/surr-high-transmission.pdf}
    \end{subfigure}
    }
    \noindent\makebox[\textwidth]{
        \begin{subfigure}[t]{0.45\textwidth}
            \centering
            % \caption{hydrogen storage}
            \includegraphics[width=\textwidth]{neardensity/surr-low-H2.pdf}
        \end{subfigure}
        \begin{subfigure}[t]{0.45\textwidth}
            \centering
        % \caption{battery storage}
        \includegraphics[width=\textwidth]{neardensity/surr-low-battery.pdf}
    \end{subfigure}
    }
    \caption{Lorem ipsum dolor sit amet, consetetur sadipscing elitr, sed diam nonumy eirmod tempor invidunt ut labore et dolore magna aliquyam erat, sed diam voluptua.}
    \label{fig:fuzzycone}
\end{figure}

\cref{fig:fuzzycone}
how:
- for each technology and cost sample, the min/max capacities for increasing epsilons form a cone; 
- the capacity ranges contained by the cone are feasible within epsilon (widens up)
- contour lines now represent the chance of being inside near-optimal cone over the whole parameter space (1e6 samples)
- this is calculated by the overlap of many cones each representing a set of cost assumptions
- due to discrete sampling points in epsilon: plots apply quadratic interpolation and gaussian filter for smoothing
interpretation:
- shows low-cost solutions common to most parameter sets (above 90\%)
- system layouts that do not meet low-cost criteria in any circumstances (below 10\%)
- the distance: the wider the contour lines are apart, the more uncertainty propagates, the closer contour lines are together the more distinct
- the height: quantifies flexibility for a given level of certainty and slack
- the angle: sensitivity towards cost slack
- the probabilities closer to optimum are smaller because little overlap
results:
- any wind:
  - extremely likely (99\%) that building 900 GW of wind capacity (in suitable locations) is attainable within 3\% of the optimum
  - building less than 600 GW has low chance of being within 8\% of the cost optimum (even in highest yield locations) - must have
  - note the nonlinearity as the space widens with increasing epsilon due to the convexity
- onshore wind and offshore wind:
  - only few solutions can forego onshore wind and remain within 8\% of the cost-optimum
  - minimisation of onshore wind will push for substitution with solar and offshore wind (see violins)
  - conversely, it is very likely possible (90\%) to build a system without offshore wind at a cost penalty of 4\% at most
  - but more offshore power generation is possible: as same chances apply for building up to 200 GW of offshore wind to remain within 8\% 
  - unlike onshore wind (where it's uncertain how little can be built), uncertainty for offshore wind deployment exists about how much can be built so that costs remain within a pre-specified range
- solar:
  - range of options within 8\% at 90\% certainty is very wide: anything between 100 GW and 1000 GW, amounts to factor 10
  - by comparison: narrow range of uncertainty about how little can be built
- transmission: 
  - the level of required transmission expansion is least affected by the cost uncertainty (sharp boundaries)
  - within 8\% it is likewise possible to plan for moderate grid reinforcement increasing capacity by 30\%
  - or to initiate extensive remodelling of the European transmission tripling the transmission volume compared to what is currently in operation
  - in any case some transmission reinforcement to balance renewable variations in space appears to be essential 
- hydrogen:
  - role: symbolise long-term storage balancing multi-week to seasonal variations
  - 100 GW is very likely within 2\% of the optimum
  - hydrogen appears to be a vital technology in many cases
  - even at 8\% slack only 25\% require no long-term storage such as hydrogen (these will be cases where battery costs are very low)
- battery
  - role: representative of short-term storage balancing daily variations
  - 90\% can work without any batteries and stay within 4\% of the optimum
  - but also 50\% can build a system within 8\% with more than 400 GW of battery
  - very wide uncertainty range, especially about how much can be built due to battery cost uncertainty

%F contour plot at fixed epsilon

\begin{figure}
    \noindent\makebox[\textwidth]{
    \begin{subfigure}[t]{0.45\textwidth}
        \centering
        \caption{wind and solar}
        \label{fig:dependencies:ws}
        \includegraphics[width=\textwidth]{dependency/2D_surr-low-wind-solar.pdf}
    \end{subfigure}
    \begin{subfigure}[t]{0.45\textwidth}
        \centering
        \caption{offshore and onshore wind}
        \label{fig:dependencies:oo}
        \includegraphics[width=\textwidth]{dependency/2D_surr-low-offwind-onwind.pdf}
    \end{subfigure}
    \begin{subfigure}[t]{0.45\textwidth}
        \centering
        \caption{hydrogen and battery storage}
        \label{fig:dependencies:hb}
        \includegraphics[width=\textwidth]{dependency/2D_surr-low-H2-battery.pdf}
    \end{subfigure}
    }
    \caption{Lorem ipsum dolor sit amet, consetetur sadipscing elitr, sed diam nonumy eirmod tempor invidunt ut labore et dolore magna aliquyam erat, sed diam voluptua.}
    \label{fig:dependencies}
\end{figure}

\cref{fig:dependencies}
how:
- cones from before assumed that the other technologies can be heavily optimised
- selected three substitutes (one can at least partially be exchanged for the other) for which most interesting trade-offs expected
- question: which combinations of wind and solar capacity, offshore and onshore turbines, and hydrogen and battery storage are likely to be achievable
- enables deriving logical and statements
- for that we examine the cross-section of fuzzy near-optimal feasible space for a given epsilon
- similar to before, the contour lines depict the overlap of the space of near-optimal alternatives across the parameter space
results:
- \cref{fig:dependencies:ws}
  - upper right: building much of both wind and solar becomes too expensive
  - bottom left: building too little of any wind or solar does not suffice to produce enough electicity
  - example: very high chance that building 1000 GW and 400 GW of solar is within 6\%
  - example: building less than 200 GW of solar and 600 GW of wind is unlikely to yield a low-cost solution
  - overall: even considering combinations of wind and solar, a wide space of low-cost options exists with moderate to high likelyhoods, but naturally range of alternatives is more constrained than when investigating individual technology
  - minimising the capacity of both primary energy providers results in (characteristics of spatial allocation of capacities will differ!)
    - location at high yield locations even if additional network expansion is necessary
    - increased preference for efficient energy storage (battery)
- \cref{fig:dependencies:oo}
  - again upper right: too much of both
  - again bottom left: too little wind of any kind
  - most certain area: building more than 700 GW onshore wind and less than 250 GW offshore wind
  - but there are some solutions with a high substitutability between onshore and offshore wind (upper left of the contour plot)
  - compared to wind and solar, the near-optimal space is more constrained
- \cref{fig:dependencies:hb}
  - some storage is definitely needed, 50 GW of each at least, more to be more certain 
  - summit: 150 GW of each

\begin{figure}
    \noindent\makebox[\textwidth]{
    \begin{subfigure}[t]{0.65\textwidth}
     \centering
     \caption{minimal onshore wind with 8\% system cost slack}
     \label{fig:nearviolin:onwind}
     \includegraphics[width=\textwidth, trim=0cm .3cm 3.3cm .63cm, clip]{violins/violin-capacities-high-prediction-min-onwind-0.08.pdf}   
    \end{subfigure}
    \begin{subfigure}[t]{0.65\textwidth}
     \centering
     \caption{minimal transmission expansion with 8\% system cost slack}
     \label{fig:nearviolin:transmission}
     \includegraphics[width=\textwidth, trim=0cm .3cm 3.3cm  .63cm, clip]{violins/violin-capacities-high-prediction-min-transmission-0.08.pdf}   
    \end{subfigure}
    }
    \caption{Lorem ipsum dolor sit amet, consetetur sadipscing elitr, sed diam nonumy eirmod tempor invidunt ut labore et dolore magna aliquyam erat, sed diam voluptua.}
    \label{fig:nearviolin}
\end{figure}

\cref{fig:nearviolin}
motivation:
- show interdependencies: what changes does the system layout experience when reaching for extremes in one technology
- pick two examples: onshore wind and transmission expansion, because they have are often linked to social acceptance problems
results:
- \cref{fig:nearviolin:onwind}
  - reduced onshore wind generation compensated predominantly by increased offshore wind parks but also more solar generation capacity
  - the increased focus on offshore wind also leads to bigger amounts of hydrogen storage, while transmission expansion needs remain similar to least-cost solutions
  - there exist cases which still require lots of onshore wind (800 GW)
- \cref{fig:nearviolin:transmission}
  - still some transmission expansion between 10\% and 30\% more
  - more generation capacity overall (likely more curtailment, more distributed resources with weaker capacity factors)
  - more hydrogen storage (compensating balancing in space with balancing in time)

\subsection{Model Validation}
%%\blindtext

%F errors vs number of samples
%F errors vs polynomial order

\begin{figure}
    \noindent\makebox[\textwidth]{
        \begin{subfigure}[t]{1.4\textwidth}
            \caption{number of samples}
            \label{fig:error:samples}
            \includegraphics[height=.2\textwidth, trim=0cm 0cm 4.5cm 0cm, clip]{error/error-r2-vs-samples-order-3-sklearn.pdf}
            \includegraphics[height=.2\textwidth, trim=0cm 0cm 4.5cm 0cm, clip]{error/error-mape-vs-samples-order-3-sklearn.pdf}
            \includegraphics[height=.2\textwidth, trim=0cm 0cm 4.5cm 0cm, clip]{error/error-mae-vs-samples-order-3-sklearn.pdf}
            \includegraphics[height=.2\textwidth]{error/error-rmse-vs-samples-order-3-sklearn.pdf}
        \end{subfigure}
    }
    \noindent\makebox[\textwidth]{
        \begin{subfigure}[t]{1.4\textwidth}
            \caption{polynomial order}
            \label{fig:error:poly}
            \includegraphics[height=.2\textwidth, trim=0cm 0cm 4.5cm 0cm, clip]{error/error-r2-vs-order-sklearn.pdf}
            \includegraphics[height=.2\textwidth, trim=0cm 0cm 4.5cm 0cm, clip]{error/error-mape-vs-order-sklearn.pdf}
            \includegraphics[height=.2\textwidth, trim=0cm 0cm 4.5cm 0cm, clip]{error/error-mae-vs-order-sklearn.pdf}
            \includegraphics[height=.2\textwidth]{error/error-rmse-vs-order-sklearn.pdf}
        \end{subfigure}
    }
    \caption{Lorem ipsum dolor sit amet, consetetur sadipscing elitr, sed diam nonumy eirmod tempor invidunt ut labore et dolore magna aliquyam erat, sed diam voluptua.}
    \label{fig:error}
\end{figure}

using cross-validation techniques 
- training set of known data
- validation/test set of unknown data to surrogate \cite{gratiet_metamodel-based_2015}
- based on low-fidelity models only
- too few samples for error analysis for high-fidelity model
- out of 500 cost samples, 100 have not been used for building/training the surrogate model; this is our test set.

error estimation, measures of accuracy:
- R2: extent of variance captured by regression quality
- MAE: to see absolute deviations
- MAPE: relative deviations
- RMSE:
- benchmark: model cross-validation error below 5\% \cite{trondle_trade-offs_2020}

\cref{fig:error}
- \cref{fig:error:samples}
  - thanks to the regularisation term, already acceptable results with as little as 50 samples
  - no significant approvement after 150 samples
  - achieve average relative errors of less than 4\%
  - (exceptions are offshore wind and battery for which the relative measure is distorted because frequently these technologies are not used at all)
  - overall the prediction of tsc is remarkably excellent (negligible errors)
  - determination coefficients above 0.95 showing that the surrogate model captures the output variance very well
  - (exception is transmission -- why?)
- \cref{fig:error:poly}
  - order of 2 and below is too simple (underfitting)
  - order of 4 and above yields no improvement (overfitting, were it not for the reqularization term)
  - as higher order requires more samples, in the interest of computational burden, order of 3 appears to be a suitable compromise




\subsection{Critical Appraisal and Outlook}
%%\blindtext

Reconstruction and Disaggregation
- When using the surrogate model we do not get the full original model outputs.
- If we are interested in particular outcomes, we can reconstruct these selectively
by adding the aggregate outcomes as additional constraints to the full model.

no sector-coupling

no path dependencies and multi-period investments

endogenous learning
- These cheap areas should then be targeted with policy (e.g. drive technology learning by subsidies)
- smoothly leads up to future work with endogenous learning in multi-horizon investment planning!
- embed technological learning in optimisation \cite{heuberger_power_2017} \cite{lopion_cost_2019}, remaining uncertainty of learning rate, but "learning by doing" included

further uncertain input parameters:
- rountrip efficiency of hydrogen storage

richer set of technologies?
- solar thermal, nuclear, CAES, separate hydrogen storage to fuel cell, biomass, geothermal, electrolysis, gas turbine
