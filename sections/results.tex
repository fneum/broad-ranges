In this section,

\subsection{Least-Cost Solutions}
%%\blindtext

%F violin plots
\begin{SCfigure}
    \includegraphics[width=0.75\textwidth]{violins/violin-capacities-high-prediction.pdf}
    \caption{
      Distribution of total system cost, generation, storage, and transmission capacities
      for least-cost solutions.
    }
    \label{fig:violin}
\end{SCfigure}

Before extending the uncertainty analysis to near-optimal solutions,
we 
\cref{fig:violin}
propagation of input uncertainties into outputs (when system cost is minimised)
tsc:
- total system cost range between 160 and 220
- means most pessimistic cost is about 40\% more expensive than using optimistic cost
- bell-like probability distribution
generation:
- all solutions build at least 350 GW of solar and 600 GW of wind
- but no more than 1100 GW of wind
- wind tendency towards higher values, solar towards lower values
- least-cost solutions prefer onshore over offshore (most solutions beyond 800 GW onshore, less than 100 GW offshore)
- but alongside battery storage capacity onshore wind capacity has the highest uncertainty range
storage:
- no least-cost solutions without hydrogen, but some without battery
- hydrogen tends to dominate battery storage, except when battery storage is very cheap
- battery also has wider uncertainty range
transmission:
- least affected by cost uncertainties
- almost doubling in relation to today's capacities
- relate to TYNDP

criticise that
- interpretation of boundaries is limited
- are not robust when looking beyond least-cost solution and marginal cost penalties are allowed
- does not tell which input uncertainties are responsible (sensitivity analysis)

%F sensitivity of built capacities

\begin{figure}
    \begin{subfigure}[t]{0.32\textwidth}
        \caption{onshore wind}
        \includegraphics[width=\textwidth]{1D/1D-onwind-onwind-high.pdf}
    \end{subfigure}
    \begin{subfigure}[t]{0.32\textwidth}
        \caption{offshore wind}
        \includegraphics[width=\textwidth]{1D/1D-offwind-offwind-high.pdf}
    \end{subfigure}
    \begin{subfigure}[t]{0.32\textwidth}
        \caption{solar}
        \includegraphics[width=\textwidth]{1D/1D-solar-solar-high.pdf}
    \end{subfigure} \\
    \begin{subfigure}[t]{0.32\textwidth}
        \caption{battery storage}
        \includegraphics[width=\textwidth]{1D/1D-battery-battery-high.pdf}
    \end{subfigure}
    \begin{subfigure}[t]{0.32\textwidth}
        \caption{hydrogen storage}
        \includegraphics[width=\textwidth]{1D/1D-H2-H2-high.pdf}
    \end{subfigure}
    \begin{subfigure}[t]{0.32\textwidth}
        \caption{transmission}
        \includegraphics[width=\textwidth]{1D/1D-transmission-H2-high.pdf}
    \end{subfigure}
    \caption{
      Sensitivity of capacities towards their own technology cost.
      The median (Q50) alongside the 5\%, 25\%, 75\%, and 95\% quantiles (Q5--Q95) display
      the sensitivity subject to the uncertainty induced by other cost parameters.
    }
    \label{fig:sensitivity}
\end{figure}

\cref{fig:sensitivity}
how:
- like local sensitivity analysis, reduce one dimension from uncertainty space
- sweep the parameter space in one dimension and obserse corresponding capacity
- for instance onshore cost on onshore capacity (presumed highest sensitivity)
- the remaining uncertainty (from other cost assumptions) is depicted by percentiles
- the overall tendency is trivial: the cheaper a technology gets, the more it is built
- but changes of slope and effects on uncertainty range across the parameter space are interesting!
result:
- slope:
  - battery  becomes economically significantly more attractive (see change of slope) once annuity falls below 75 EUR/kW/a
  - [translate into an overnight investment with exemplary lifetime/interest] (regardless of other cost uncertainties), below it is a minor to hydrogen
  - hydrogen on the other hand - steady slope and uncertainty band
- uncertainty:
  - low cost of onshore wind - much onshore wind with low uncertainty
  - high cost of onshore wind - how much is built greatly depends on other cost parameters
  - the opposite for offshore wind (and solar)
- other:
  - the limited uncertainty about optimal levels of network expansion is mostly due to hydrogen storage cost. As cost of storing H2 falls, less grid reinforcment chosen (can also see this  in Sobol indices)
transition:
- these self-sensitivities are only a selection
- we can formalise how input uncertainties affect each outcome with sensitivity indices
- next Sobol

%F sobol indices

\begin{figure}
    \begin{subfigure}[t]{0.45\textwidth}
        \caption{first-order Sobol indices [\%]}
        \label{fig:sobol:first}
        \includegraphics[width=\textwidth]{sobol/sobol-m-high.pdf}
    \end{subfigure}
    \begin{subfigure}[t]{0.54\textwidth}
        \caption{total Sobol indices [\%]}
        \label{fig:sobol:total}
        \includegraphics[width=\textwidth]{sobol/sobol-t-high-bar.pdf}
    \end{subfigure}
    \caption[First-order and total Sobol indices]{
      Sobol indices. These sensitivity indices attribute output variance to random input variables
      and reveal which inputs the outputs are most sensitive to. The first-order Sobol indices
      quantify the share of output variance due to variations in one input parameter alone.
      The total Sobol indices further include interactions with other input variables.
      Total Sobol indices can be greater than 100\% if the contributions are not purely additive.
    }
    \label{fig:sobol}
\end{figure}

Sobol / sensitivity indices:
- decomposition of output variance and attribution to random input variables
- separate influential from non-influential parameters
- The polynomials can be used for calculating Sobol indices efficiently in post-processing \cite{sudret_global_2008}
- used in \cite{trondle_trade-offs_2020,mavromatidis_uncertainty_2018}
- using chaospy \cite{feinberg_chaospy_2015}
- first-order Sobol: share of output variance due to variations in one input parameter alone (averaged over variations in other input parameters)
- total Sobol indices: contribution to output variance including all interactions with other input variables (can be greater than 100\% if not purely additive)

\cref{fig:sobol} \cref{fig:sobol:first} \cref{fig:sobol:total}
- two ways to display as a heatmap or as stacked bar chart
- first-order effects dominate, but there are second-order effects
- tsc largely decided by how expensive it is to build onshore wind capacity, followed by cost of hydrogen storage
- wind deploymend almost exclusively governed by cost of onwind/offwind
- other carriers yield a more varied picture (depend on other technologies)
- solar deployment also depends on onshore wind and battery cost
- amount of hydrogen influenced by battery and H2 cost alike
- The Sobol indices of the least-cost solution are widely shared across the near-optimal solutions with little deviation.

\subsection{Near-Optimal Solutions}
%%\blindtext

- uncertainty quantification of outputs at least-cost solutions: check
- sensitivity analysis of inputs on outputs at least-cost solutions: check
- has been shown: for single-cost parameter set a wide array of technologically diverse but similarly costly solutions exist
- next we want to examine where agree/disagree, feasible alternatives common to all, few or no parameter sets
- robust conclusions on near-optimal solutions that are not affected by uncertainty: next
- in other words: find solutions that are likely within epsilon of the least-cost solution
- combination of cost uncertainty and slack leads to: look at solutions between 160 (optimistic least-cost) and 240 (pessimistic with maximum slack) bn EUR pa

%F fuzzy c-shaped plots

\begin{figure}
    \vspace{-2cm}
    \noindent\makebox[\textwidth]{
    \begin{subfigure}[t]{0.45\textwidth}
        \centering
        % \caption{any wind}
        \includegraphics[width=\textwidth]{neardensity/surr-high-wind.pdf}
    \end{subfigure}
    \begin{subfigure}[t]{0.45\textwidth}
        \centering
        % \caption{onshore wind}
        \includegraphics[width=\textwidth]{neardensity/surr-high-onwind.pdf}
    \end{subfigure}
    \begin{subfigure}[t]{0.45\textwidth}
        \centering
        % \caption{offshore wind}
        \includegraphics[width=\textwidth]{neardensity/surr-high-offwind.pdf}
    \end{subfigure}
    }
    \noindent\makebox[\textwidth]{
    \begin{subfigure}[t]{0.45\textwidth}
        \centering
        % \caption{solar}
        \includegraphics[width=\textwidth]{neardensity/surr-high-solar.pdf}
    \end{subfigure}
    \begin{subfigure}[t]{0.45\textwidth}
        \centering
        % \caption{transmission network}
        \includegraphics[width=\textwidth]{neardensity/surr-high-transmission.pdf}
    \end{subfigure}
    }
    \noindent\makebox[\textwidth]{
        \begin{subfigure}[t]{0.45\textwidth}
            \centering
            % \caption{hydrogen storage}
            \includegraphics[width=\textwidth]{neardensity/surr-low-H2.pdf}
        \end{subfigure}
        \begin{subfigure}[t]{0.45\textwidth}
            \centering
        % \caption{battery storage}
        \includegraphics[width=\textwidth]{neardensity/surr-low-battery.pdf}
    \end{subfigure}
    }
    \caption{
    Space of near-optimal solutions by technology under cost uncertainty.
    For each technology and cost sample,
    the minimum and maximum capacities obtained for increasing cost penalties
    $\varepsilon$ form a cone, starting from a common least-cost solution. 
    By arguments of convexity, the capacity ranges contained by the cone can be near-optimal and and feasible, given a degree of freedom in the other technologies.
    From optimisation theory, we know that the cones widen up for increased slacks.
    As we consider technology cost uncertainty, the cone will look slightly different for each sample.
    The contour lines represent the frequency a solution is inside near-optimal cone over the whole parameter space.
    This is calculated from the overlap of many cones, each representing a set of cost assumptions.
    Due to discrete sampling points in the $\varepsilon$-dimension, the plots further apply quadratic interpolation and gently a Gaussian filter for smoothing.
    }
    \label{fig:fuzzycone}
\end{figure}

\cref{fig:fuzzycone} depicts low-cost solutions common to most parameter sets (e.g. above 90\% contour)
as well as system layouts that do not meet low-cost criteria in any circumstances.
The wider the contour lines are apart, the more uncertainty exists about the boundaries.
The closer contour lines are together, the more specific the limits are.
The height of the quantiles quantifies flexibility for a given level of certainty and slack;
the angle presents information about the sensitivity towards cost slack.
% - the probabilities closer to optimum are smaller because little overlap

From \cref{fig:fuzzycone} we can see that it is 
highly likely that building 900 GW of wind capacity is within 3\% of the optimum, and that
conversely building less than 600 GW has a low chance of being near the cost optimum.
Only a few solutions can forego onshore wind entirely and remain within 8\% of the cost-optimum,
whereas it is very likely possible to build a system without offshore wind at a cost penalty of 4\% at most.
On the other hand, more offshore wind generation is equally possible.
Unlike for onshore wind, where it is more uncertain how little can be built,
uncertainty regarding offshore wind deployment exists about how much can be built
so that costs remain within a pre-specified range.
For solar, the range of options within 8\% of the cost optimum at 90\% certainty is very wide.
Anything between 100 GW and 1000 GW appears feasible.
In comparison to onshore wind, the uncertainty about minimal solar requirements is narrower.
Nonetheless, the level of required transmission expansion is least affected by the cost uncertainty.
To remain within $\varepsilon=8\%$ it is just as possible to
plan for moderate grid reinforcement by 30\% as 
is initiating extensive remodelling of the grid by tripling the transmission volume
compared to what is currently in operation.
These results indicate that in any case some transmission reinforcement
to balance renewable variations across the continent appears to be essential.
Hydrogen storage, symbolising long-term storage, also gives the impression of a vital technology in many cases.
Building 100 GW of hydrogen storage capacity is likely viable within 2\% of the cost optimum 
and, even at $\varepsilon=8\%$, only 25\% of cost samples require no long-term storage; 
cases where battery costs are exceptionally low.
Overall, 90\% of cases appear to function without any short-term battery storage
while the system cost rises by 4\% at most.
However, especially battery storage exhibits a large degree of freedom to build more.

%F contour plot at fixed epsilon

\begin{figure}
    \noindent\makebox[\textwidth]{
    \begin{subfigure}[t]{0.45\textwidth}
        \centering
        \caption{wind and solar}
        \label{fig:dependencies:ws}
        \includegraphics[width=\textwidth]{dependency/2D_surr-low-wind-solar.pdf}
    \end{subfigure}
    \begin{subfigure}[t]{0.45\textwidth}
        \centering
        \caption{offshore and onshore wind}
        \label{fig:dependencies:oo}
        \includegraphics[width=\textwidth]{dependency/2D_surr-low-offwind-onwind.pdf}
    \end{subfigure}
    \begin{subfigure}[t]{0.45\textwidth}
        \centering
        \caption{hydrogen and battery storage}
        \label{fig:dependencies:hb}
        \includegraphics[width=\textwidth]{dependency/2D_surr-low-H2-battery.pdf}
    \end{subfigure}
    }
    \caption{
      Space of near-optimal solutions by selected pairs of technologies under cost uncertainty.
    Just like in \cref{fig:fuzzycone}, the contour lines depict the overlap of the space of near-optimal alternatives across the parameter space.
    It can be thought of as the cross-section of the probabilistic near-optimal feasible space for a given $\varepsilon$
    in two technology dimensions and highlights that the extremes of two technologies from \cref{fig:fuzzycone} cannot be achieved simultaneously.
    }
    \label{fig:dependencies}
\end{figure}

The fuzzy cones from \cref{fig:fuzzycone} assume that the other technologies can be heavily optimised.
But as there are dependencies between the technologies, in \cref{fig:dependencies}
we furthermore present the probabilistic near-optimal space for three selected technology
pairs at fixed $\varepsilon=6\%$ for which we expect the most interesting trade-offs.
It addresses the question about which combinations of wind and solar capacity,
offshore and onshore turbines, and hydrogen and battery storage are likely to be achievable.

First, \cref{fig:dependencies:ws} addresses constraints between wind and solar.
The upper right boundary exists because building much of both wind and solar would bee too expensive.
The absence of solutions in the bottom left corner means that
building too little of any wind or solar does not suffice to produce enough electricity.
From the shape and contours, we see a high chance
that building 1000 GW of wind and 400 GW of solar is within 6\% of the cost-optimum.
On the other hand, building less than 200 GW of solar and 600 GW of wind is unlikely to yield a low-cost solution.
Minimising the capacity of both primal energy sources will shift capacity installations to
high-yield locations even if additional network expansion is necessary and boost
the preference for highly efficient storage technologies.
Overall, even considering combinations of wind and solar,
a wide space of low-cost options exists with moderate to high likelihood,
although the range of alternatives is shown to be more constrained.

\cref{fig:dependencies:oo} concerns the trade-off between onshore wind and offshore wind.
The most certain area is characterised by building more than 600 GW onshore wind,
and less than 250 GW offshore wind capacity.
However, there are some solutions with high substitutability between onshore and offshore wind,
shown in the upper left of the contour plot.
Compared to wind and solar, the range of near-optimal solutions is more constrained.
Finally, \cref{fig:dependencies:hb} underlines that some storage is definitely needed,
50 GW of each at least in any case. Highest likelihoods are attained when building 150 GW of each.

\begin{figure}
    \noindent\makebox[\textwidth]{
    \begin{subfigure}[t]{0.65\textwidth}
     \centering
     \caption{minimal onshore wind with 8\% system cost slack}
     \label{fig:nearviolin:onwind}
     \includegraphics[width=\textwidth, trim=0cm .3cm 3.3cm .63cm, clip]{violins/violin-capacities-high-prediction-min-onwind-0.08.pdf}   
    \end{subfigure}
    \begin{subfigure}[t]{0.65\textwidth}
     \centering
     \caption{minimal transmission expansion with 8\% system cost slack}
     \label{fig:nearviolin:transmission}
     \includegraphics[width=\textwidth, trim=0cm .3cm 3.3cm  .63cm, clip]{violins/violin-capacities-high-prediction-min-transmission-0.08.pdf}   
    \end{subfigure}
    }
    \caption{
      Distribution of total system cost, generation, storage, and transmission capacities
      for two near-optimal search directions with $\varepsilon=8\%$ system cost slack.
    }
    \label{fig:nearviolin}
\end{figure}

Neither of the aforementioned contour plots (\crefrange{fig:fuzzycone}{fig:dependencies}) expose
what changes the system layout experiences when reaching for the extremes in one technology.
Therefore, we show in \cref{fig:nearviolin} how the system-wide capacity distributions vary 
compared to the least-cost solutions (\cref{fig:violin}) for two exemplary alternative objectives.
For that, we chose minimising onshore wind capacity and transmission expansion
because they are often linked to public opposition.

\cref{fig:nearviolin:onwind} illustrates that reducing onshore wind capacity is
predominantly compensated by increased offshore wind generation but also added solar capacities.
The increased focus on offshore wind also leads to a tendency towards more hydrogen storage,
while transmission expansion levels are similarly distributed as for the least-cost solutions.
From \cref{fig:nearviolin:transmission} we can further extract that avoiding transmission expansion entails
more hydrogen storage that compensates balancing in space with balancing in time,
and more generation capacity overall, where resources are distributed to locations with
high demand but weaker capacity factors and more heavily curtailed.

\subsection{Critical Appraisal and Outlook}
%%\blindtext

Reconstruction and Disaggregation
- When using the surrogate model we do not get the full original model outputs.
- If we are interested in spatially explicit outcomes, we can either take one of the samples
or reconstruct selectively by adding the aggregate outcomes as additional constraints to the full model.

no sector-coupling

no path dependencies and multi-period investments

endogenous learning
- These cheap areas should then be targeted with policy (e.g. drive technology learning by subsidies)
- smoothly leads up to future work with endogenous learning in multi-horizon investment planning!
- embed technological learning in optimisation \cite{heuberger_power_2017} \cite{lopion_cost_2019}, remaining uncertainty of learning rate, but "learning by doing" included

further uncertain input parameters:
- rountrip efficiency of hydrogen storage

richer set of technologies?
- solar thermal, nuclear, CAES, separate hydrogen storage to fuel cell, biomass, geothermal, electrolysis, gas turbine
