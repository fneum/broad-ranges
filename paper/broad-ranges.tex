\documentclass[1p,11pt]{elsarticle}

\journal{Joule}

\widowpenalty10000
\clubpenalty10000

%% `Elsevier LaTeX' style
\bibliographystyle{elsarticle-num}
\biboptions{numbers,sort&compress,super}
\abstracttitle{Summary}

% format hacks
\usepackage{libertine}
\usepackage{libertinust1math}
% \renewcommand*\familydefault{\sfdefault} % biolinum
% \usepackage[margin=3.5cm]{geometry}

% \usepackage[latin1]{inputenc}
% \usepackage{amsfonts}
% \usepackage{amssymb}

\usepackage{amsmath}
\usepackage{bbold}
\usepackage{graphicx}
\usepackage{eurosym}
\usepackage{mathtools}
\usepackage{url}
\usepackage{booktabs}
\usepackage{epstopdf}
\usepackage{xfrac}
\usepackage{bm}
\usepackage[colorlinks]{hyperref}
\usepackage[nameinlink,sort&compress,capitalise,noabbrev]{cleveref}
\usepackage[leftcaption,raggedright]{sidecap}
\usepackage{subcaption}
\usepackage{blindtext}

\usepackage[parfill]{parskip}

\graphicspath{{../notebooks/2021-11-graphics/}}

% \usepackage[
% 	type={CC},
% 	modifier={by},
% 	version={4.0},
% ]{doclicense}

\newcommand{\abs}[1]{\left|#1\right|}
\newcommand{\norm}[1]{\left\lVert#1\right\rVert}
\newcommand{\set}[1]{\left\{#1\right\}}
\DeclareMathOperator*{\argmin}{\arg\!\min}
\def\cL{\mathcal{L}}
\def\cN{\mathcal{N}}
\def\cT{\mathcal{T}}
\def\cM{\mathcal{M}}
\def\cA{\mathcal{A}}
\def\ba{{\bm{\alpha}}}
\def\card{\text{card}\,}
\def\x{\boldsymbol{\mathsf{x}}}


\newcommand{\usection}[2]{
	\section*{#1}
	\label{#2}
	\addcontentsline{toc}{section}{\nameref{#2}}
}

\newcommand{\usubsection}[2]{
	\subsection*{#1}
	\label{#2}
	\addcontentsline{toc}{subsection}{\nameref{#2}}
}

\usepackage[prependcaption,textsize=scriptsize]{todonotes}

% \usepackage{lineno}

\usepackage{orcidlink}

\urlstyle{sf}

\begin{document}

\begin{frontmatter}

	\title{Broad Ranges of Investment Configurations for Renewable Power
	Systems, Robust to Cost Uncertainty and Near-Optimality}

	\author[tubaddress,kitaddress]{Fabian Neumann\,\orcidlink{0000-0001-8551-1480}\corref{correspondingauthor}}
	\ead{f.neumann@tu-berlin.de}
	\author[tubaddress,kitaddress]{Tom Brown\,\orcidlink{0000-0001-5898-1911}}
	\cortext[correspondingauthor]{Corresponding author}
	\address[tubaddress]{Department of Digital Transformation in Energy Systems, Institute of Energy Technology, Technische Universität Berlin (TUB), Einsteinufer 25 (TA 8), 10587, Berlin, Germany}
	\address[kitaddress]{Institute for Automation and Applied Informatics (IAI), Karlsruhe Institute of Technology (KIT), Hermann-von-Helmholtz-Platz 1, 76344, Eggenstein-Leopoldshafen, Germany}

	\begin{abstract}
		% Nature Energy 150 words

TODO
	\end{abstract}

	\begin{keyword}
		renewable power system, capacity expansion planning, modelling-to-generate-alternatives, uncertainty quantification, global sensitivity analysis, multi-objective optimisation
	\end{keyword}

\end{frontmatter}

% \linenumbers

% \addcontentsline{toc}{section}{Highlights}
% \section*{Highlights}

% \begin{itemize}
% 	\item none
% 	\item none
% 	\item none
% \end{itemize}

\newpage

\newgeometry{top=2cm, bottom=3cm}

\par\noindent\rule{\textwidth}{0.4pt}

\section*{Graphical Abstract}

\includegraphics[width=\textwidth]{graphics/graphical-abstract.png}

\par\noindent\rule{\textwidth}{0.4pt}

\section*{Highlights}

\begin{itemize}
	\item uses open electricity system model PyPSA-Eur to explore near-optimal trade-offs
	\item wide range of options to design renewable power systems within 8\% of optimum
	\item many of these options are robust to uncertain technology cost developments
	\item few boundary conditions exist: some long-term storage and network expansion
	\item degrees of freedom can help policymakers to circumvent public acceptance issues
\end{itemize}

\par\noindent\rule{\textwidth}{0.4pt}

\section*{Context \& Scale}

We address the perils of narrowly following optimisation results in renewable
electricity system planning by presenting a wide range of almost equally
cost-effective system designs that are technologically diverse and robust to
uncertain technology cost projections. Especially along dimensions that affect
levels of social acceptance, like the development of new overhead power lines or
onshore wind turbines, we want to improve our understanding of what actions are
likely viable within given cost ranges. We present system designs with a high
chance of involving a limited cost increase, just as we identify those that are
unlikely to be cost-efficient. Knowledge of the vast degrees of freedom we find
in our analysis is highly policy-relevant. Policymakers can leverage these weak
trade-offs to circumvent building infrastructure for which there is limited
public support.
% \par\noindent\rule{\textwidth}{0.4pt}

% \section*{Words}
% 3278 (excl. preamble, captions, and \nameref{sec:methods})

\restoregeometry

\newpage
\usection{Introduction}{sec:introduction}

% energy system modelling challenges

Energy system models have become a pivotal instrument for policy-making
to find cost-efficient system layouts that satisfy ambitious targets for climate change mitigation.
But even though they have proliferated in spatial, temporal, technological and sectoral detail and scope in recent years,
least-cost optimisation models can easily give a false sense of exactness.
For example, they involve the danger of underplaying social factors
not contained by the model formulation, such as
rising public opposition to certain technologies like transmission, wind, nuclear, or biomass
or an uneven distribution of infrastructure \cite{mccollum_energy_2020,sasse_regional_2020}.

Presenting just a single least-cost solution for a single set of cost assumptions
not only underplays uncertainties inherent to technology cost projections the
investment planning models are are known to be very sensitive to
\cite{trondle_trade-offs_2020,Pfenninger2014,yue_review_2018,pye_assessing_2018,pye_modelling_2020},
but also neglects a wide variety of alternative solutions
that are equally feasible and only marginally more expensive
\cite{nearoptimal,lombardi_policy_2020,sasse_distributional_2019}.

% why are alternative solutions important

Such alternatives arising by deviating from the least-cost solution are attractive for policymakers,
because they allow them to make decisions based on other, non-economic criteria,
without affecting the cost-effectiveness of the system.
In other words, knowing that many similarly costly but technologically diverse solutions
exist helps to accomodate political and social dimensions that are otherwise hard to quantify.
Additionally acknowledging sensitivities towards cost assumptions further raises confidence
in modelling results. If we can understand the key drivers of the cost structure and
technology preferences \cite{usher_value_2015,moret_characterization_2017},
central parameters can be targeted for technological learning with policy.

% parametric uncertainty

% - technology cost, useable potentials, discount rate, weather year, climate change, projection and regional distribution electricity demand to only name a few

Numerous examples of parametric uncertainty analysis:
assessed from different angles:
- which parameters are uncertain and to what extent
- what parameters influence outputs the most
- how uncertainty propagates to the outputs outputs
- how to hedge against uncertainty with stochastic or robust optimisation
Karush-Kuhn-Tucker multipliers
can tell us about the local sensitivity to small changes in the constraints at the optimal point,
but not the global behaviour on the feasible space, which may be highly non-linear.
For instance, these dual variables translate a carbon cap to a carbon price, map a renewable generation target to a required subsidy, limited potentials to a scarcity cost, or even output nodal prices \cite{marketvalue}.
While in scenario-based sensitivity analysis only
a handful of potentially biased narratives are developed \cite{DeCarolis2017,soroudi_decision_2013,usher_value_2015}, local sensitivity analysis sweeps the parameter space
across costs and constraint constants altering only one dimension at a time \cite{schlachtberger_cost_2018}.
Popular target parameters include the investement cost and potentials \cite{schlachtberger_cost_2018},
transmission expansion limits \cite{schlachtberger_benefits_2017},
reference weather years and climate change \cite{bloomfield_2021,impactofclimatechange},
distributional equity requirements \cite{sasse_regional_2020,sasse_distributional_2019},
or varying emission reduction targets.
But given that only a small fraction of the uncertainty space
is sampled by single-parameter variation alone,
decisive interactions may be overlooked \cite{pizarro-alonso_uncertainties_2019}.

global sensitivity analysis
- vary multiple paramters at a time
- analyses how each random input variable affects the variance of outputs \cite{sudret_global_2008}
- high computational requirements \cite{usher_value_2015,pizarro-alonso_uncertainties_2019,moret_robust_2016}
- applications: \cite{trondle_trade-offs_2020,fais_impact_2016,mavromatidis_uncertainty_2018,pilpola_analyzing_2020}
although account for uncertainty, not compromises and trade-offs

% structural uncertainty

Besides parametric uncertainty, there is also a component structural uncertainties of the
optimisation model.
Examples :
- power flow approximation \cite{flowandlosses}, spatial aggregation \cite{hoersch_spatial_2017}, temporal aggregation \cite{kotzur_tsa_2018}, technology simplification \cite{DeCarolis2017}, or--as will be studied in this contribution--degrees of freedom near the cost-optimum \cite{nearoptimal}.
Techniques like multi-objective optimisation and modelling-to-generate-alternatives
- epsilon constraint method used in \cite{nearoptimal}
- determine Pareto-fronts \cite{mavrotas_effective_2009}
- find near-optimal solutions
- applied in for integrated assessment models \cite{Price2017}, 
for the United States electricity sector \cite{DeCarolis2016},
the Italian power system \cite{lombardi_policy_2020},
a single node sector-coupling model of Germany \cite{nacken_integrated_2019},
and combined with a quickhull algorithm to span a polytope of low-cost solutions \cite{pedersen_modeling_2020}.

%structural AND parametric uncertainty

Previous works have mostly put focus on either near-optimal or parametric uncertainty
- both have an effect on confidence in conclusions
- related: least-cost solutions for one parameter sample may be within 5\% of the optimum for another parameter sample, or vice versa
- only few: MGA including cost uncertainty \cite{Trutnevyte2013,Li2017,lombardi_policy_2020}

% this work / novelty

In this contribution, we systematically explore parameter space of technology cost projections and
investigate how they affect the space of near-optimal solutions
of a fully renewable European electricity system model
that is both spatially and temporally resolved. % open source
That means we solve long-term investment planning problems that
coordinate generation, transmission and storage investments,
while accounting for a wide range of cost projections for wind, solar, battery and
hydrogen storage technologies.
To handle the computational burden incurred by searching for near-optimal solutions
alongside evaluating many different cost parameter sets,
we employ multi-fidelity surrogate modelling techniques based on sparse polynomial chaos expansion
that allow us to merge results from a simpler and a more detailed model.
As we solve more than 50,000 resource-intensive optimisation problems,
this kind of analysis has only become possible by the low-cost availability 
of high-performance computing resources. 

% Two ways:
% - expand distribution view of cost-optimal capacities by the space of near-optimal solutions
% - or vice versa: augment the near-optimal solution space by adding variance induced by cost uncertainty

% results

We investigate alternative solutions beyond least-cost
that have a high chance of involving a limited cost increase,
just as we identify regions that are unlikely to be cost-efficient.
We derive both, ranges of options and technology-specific boundary conditions,
that are not affected by cost uncertainty and must be 
met to keep the total system cost within a specified range.
Our results show that many such similarly costly,
but technologically diverse solutions exist regardless
how technology cost developments will unfold.



In this section, we first quantify the uncertainty of least-cost solutions incurred by unknown
future technology cost
and conduct a global sensitivity analysis to set a foundation.
We then expand the uncertainty analysis to the space of nearly cost-optimal solutions,
which yields us insights about the consistency of alternatives across a variety of cost parameters.
Finally, the results are critically discussed.

\subsection{Least-Cost Solutions}
%%\blindtext

%F violin plots
\begin{SCfigure}
    \includegraphics[width=0.75\textwidth]{violins/violin-capacities-high-prediction.pdf}
    \caption{
      Distribution of total system cost, generation, storage, and transmission capacities
      for least-cost solutions.
    }
    \label{fig:violin}
\end{SCfigure}

We approach the uncertainty analysis to near-optimal solutions
by reviewing the propagation of input uncertainties into 
least-cost solutions first and expanding gradually from there.
As displayed in \cref{fig:violin},
the total annual system costs vary between 160 and 220 billion Euro per year.
This means the most pessimistic cost projections
entail about 40\% higher cost than the most optimistic projections.
All least-cost solutions build at least 350 GW solar and 600 GW wind, but no more than 1100 GW.
While wind capacities tend towards higher values, solar capacities tend towards lower values.
We observe that least-cost solutions clearly prefer onshore over offshore wind, yet
alongside battery storage onshore wind features the highest uncertainty range.
The cost optimum gravitates towards hydrogen storage rather than battery storage,
unless battery storage becomes very cheap.
There are no least-cost solutions without hydrogen, only some without battery storage.
Transmission network expansion is least affected by cost uncertainty and consistently
doubled compared to today's capacities.
However, the interpretation of the observed ranges is limited because
they are not robust when we look beyond the least-cost solutions and allow marginal cost penalties.
Morover, the distribution of outputs does not yet convey information about their sensitivity to cost assumptions. 

%F sensitivity of built capacities

\begin{figure}
    \begin{subfigure}[t]{0.32\textwidth}
        \caption{onshore wind}
        \includegraphics[width=\textwidth]{1D/1D-onwind-onwind-high.pdf}
    \end{subfigure}
    \begin{subfigure}[t]{0.32\textwidth}
        \caption{offshore wind}
        \includegraphics[width=\textwidth]{1D/1D-offwind-offwind-high.pdf}
    \end{subfigure}
    \begin{subfigure}[t]{0.32\textwidth}
        \caption{solar}
        \includegraphics[width=\textwidth]{1D/1D-solar-solar-high.pdf}
    \end{subfigure} \\
    \begin{subfigure}[t]{0.32\textwidth}
        \caption{battery storage}
        \includegraphics[width=\textwidth]{1D/1D-battery-battery-high.pdf}
    \end{subfigure}
    \begin{subfigure}[t]{0.32\textwidth}
        \caption{hydrogen storage}
        \includegraphics[width=\textwidth]{1D/1D-H2-H2-high.pdf}
    \end{subfigure}
    \begin{subfigure}[t]{0.32\textwidth}
        \caption{transmission}
        \includegraphics[width=\textwidth]{1D/1D-transmission-H2-high.pdf}
    \end{subfigure}
    \caption{
      Sensitivity of capacities towards their own technology cost.
      The median (Q50) alongside the 5\%, 25\%, 75\%, and 95\% quantiles (Q5--Q95) display
      the sensitivity subject to the uncertainty induced by other cost parameters.
    }
    \label{fig:sensitivity}
\end{figure}

\cref{fig:sensitivity} addresses a selection of local self-sensitivities, i.e.~how the cost of a technology influences its deployment,
while displaying the remaining uncertainty induced by other cost parameters.
The overall tendency is trivial: the cheaper a technology becomes, the more it is built.
However, changes of slope and effects on uncertainty range across the parameter space are insightful nonetheless.
For instance, \cref{fig:sensitivity} reveals that battery storage becomes significantly more 
attractive economically once its annuity falls below 75 EUR/kW/a, whereas hydrogen storage is
characterised by a steady slope.
A low cost of onshore wind brings about much onshore wind capacity with low uncertainty,
whereas if onshore wind costs are high how much is built greatly depends on other cost parameters.
The opposite is true for offshore wind and solar. 
The limited uncertainty about cost-optimal levels of grid expansion is mostly due to the cost of hydrogen storage.
As the cost of hydrogen storage falls, less grid reinforcment is chosen.
Since the presented self-sensitivities only exhibit a fraction of sensitivities,
in the next step we formalise how input uncertainties affect each outcome by 
systematically with variance-based global sensitivity analysis.

%F sobol indices

\begin{figure}
    \begin{subfigure}[t]{0.45\textwidth}
        \caption{first-order Sobol indices [\%]}
        \label{fig:sobol:first}
        \includegraphics[width=\textwidth]{sobol/sobol-m-high.pdf}
    \end{subfigure}
    \begin{subfigure}[t]{0.54\textwidth}
        \caption{total Sobol indices [\%]}
        \label{fig:sobol:total}
        \includegraphics[width=\textwidth]{sobol/sobol-t-high-bar.pdf}
    \end{subfigure}
    \caption[First-order and total Sobol indices]{
      Sobol indices. These sensitivity indices attribute output variance to random input variables
      and reveal which inputs the outputs are most sensitive to. The first-order Sobol indices
      quantify the share of output variance due to variations in one input parameter alone.
      The total Sobol indices further include interactions with other input variables.
      Total Sobol indices can be greater than 100\% if the contributions are not purely additive.
    }
    \label{fig:sobol}
\end{figure}

Sensitivity indices, or Sobol indices, attribute the observed output variance to each input.
For our application, the Sobol indices can, for instance, tell us which technology cost
contributes the most to total system cost or how much of a specific technology will be built.
The first-order Sobol indices describe the share of output variance
due to variations in one input alone averaged over variations in the other inputs.
Total Sobol indices also consider higher-order interactions,
which are greater than 100\% if the relations are not purely additive.
Sobol indices have been applied in the context of energy systems (e.g. \cite{trondle_trade-offs_2020,mavromatidis_uncertainty_2018})
and we profit from the advantage that we can conveniently compute the indices analytically
from the polynomial chaos expansion \cite{sudret_global_2008}.

\cref{fig:sobol} displays the first-order and total Sobol indices for least-cost solutions.
It shows that the total system cost is largely determined by how expensive it is to build onshore wind capacity,
followed by the cost of hydrogen storage. The amount of wind in the system is 
almost exclusively governed by the cost of onshore and offshore wind parks.
Other carriers yield a more varied picture.
The cost-optimal solar capacities additionally depend on onshore wind and battery costs.
The amount of hydrogen storage is influenced by battery and hydrogen storage cost alike.
Overall, although the first-order effects dominate, there are noticable higher-order effects,
which are largest for the amount of transmission.

\subsection{Near-Optimal Solutions}
%%\blindtext

So far we quantified the output uncertainty and analysed the sensitivity towards inputs at least-cost solutions only.
Yet, it has been previously shown that even for a single cost parameter set a wide array of
technologically diverse but similarly costly solutions exists \cite{nearoptimal}.
In the next step, we examine how technology cost uncertainty affects the outline of near-optimal solutions.  
By identifying feasible alternatives common to all, few or no cost samples, we improve the robustness of the results.

%F fuzzy c-shaped plots

\begin{figure}
    \vspace{-2cm}
    \noindent\makebox[\textwidth]{
    \begin{subfigure}[t]{0.45\textwidth}
        \centering
        % \caption{any wind}
        \includegraphics[width=\textwidth]{neardensity/surr-high-wind.pdf}
    \end{subfigure}
    \begin{subfigure}[t]{0.45\textwidth}
        \centering
        % \caption{onshore wind}
        \includegraphics[width=\textwidth]{neardensity/surr-high-onwind.pdf}
    \end{subfigure}
    \begin{subfigure}[t]{0.45\textwidth}
        \centering
        % \caption{offshore wind}
        \includegraphics[width=\textwidth]{neardensity/surr-high-offwind.pdf}
    \end{subfigure}
    }
    \noindent\makebox[\textwidth]{
    \begin{subfigure}[t]{0.45\textwidth}
        \centering
        % \caption{solar}
        \includegraphics[width=\textwidth]{neardensity/surr-high-solar.pdf}
    \end{subfigure}
    \begin{subfigure}[t]{0.45\textwidth}
        \centering
        % \caption{transmission network}
        \includegraphics[width=\textwidth]{neardensity/surr-high-transmission.pdf}
    \end{subfigure}
    }
    \noindent\makebox[\textwidth]{
        \begin{subfigure}[t]{0.45\textwidth}
            \centering
            % \caption{hydrogen storage}
            \includegraphics[width=\textwidth]{neardensity/surr-low-H2.pdf}
        \end{subfigure}
        \begin{subfigure}[t]{0.45\textwidth}
            \centering
        % \caption{battery storage}
        \includegraphics[width=\textwidth]{neardensity/surr-low-battery.pdf}
    \end{subfigure}
    }
    \caption{
    Space of near-optimal solutions by technology under cost uncertainty.
    For each technology and cost sample,
    the minimum and maximum capacities obtained for increasing cost penalties
    $\varepsilon$ form a cone, starting from a common least-cost solution. 
    By arguments of convexity, the capacity ranges contained by the cone can be near-optimal and and feasible, given a degree of freedom in the other technologies.
    From optimisation theory, we know that the cones widen up for increased slacks.
    As we consider technology cost uncertainty, the cone will look slightly different for each sample.
    The contour lines represent the frequency a solution is inside near-optimal cone over the whole parameter space.
    This is calculated from the overlap of many cones, each representing a set of cost assumptions.
    Due to discrete sampling points in the $\varepsilon$-dimension, the plots further apply quadratic interpolation and gently a Gaussian filter for smoothing.
    }
    \label{fig:fuzzycone}
\end{figure}

\cref{fig:fuzzycone} depicts low-cost solutions common to most parameter sets (e.g. above 90\% contour)
as well as system layouts that do not meet low-cost criteria in any circumstances.
The wider the contour lines are apart, the more uncertainty exists about the boundaries.
The closer contour lines are together, the more specific the limits are.
The height of the quantiles quantifies flexibility for a given level of certainty and slack;
the angle presents information about the sensitivity towards cost slack.

From \cref{fig:fuzzycone} we can see that it is 
highly likely that building 900 GW of wind capacity is within 3\% of the optimum, and that
conversely building less than 600 GW has a low chance of being near the cost optimum.
Only a few solutions can forego onshore wind entirely and remain within 8\% of the cost-optimum,
whereas it is very likely possible to build a system without offshore wind at a cost penalty of 4\% at most.
On the other hand, more offshore wind generation is equally possible.
Unlike for onshore wind, where it is more uncertain how little can be built,
uncertainty regarding offshore wind deployment exists about how much can be built
so that costs remain within a pre-specified range.
For solar, the range of options within 8\% of the cost optimum at 90\% certainty is very wide.
Anything between 100 GW and 1000 GW appears feasible.
In comparison to onshore wind, the uncertainty about minimal solar requirements is narrower.
Nonetheless, the level of required transmission expansion is least affected by the cost uncertainty.
To remain within $\varepsilon=8\%$ it is just as possible to
plan for moderate grid reinforcement by 30\% as 
is initiating extensive remodelling of the grid by tripling the transmission volume
compared to what is currently in operation.
These results indicate that in any case some transmission reinforcement
to balance renewable variations across the continent appears to be essential.
Hydrogen storage, symbolising long-term storage, also gives the impression of a vital technology in many cases.
Building 100 GW of hydrogen storage capacity is likely viable within 2\% of the cost optimum 
and, even at $\varepsilon=8\%$, only 25\% of cost samples require no long-term storage; 
cases where battery costs are exceptionally low.
Overall, 90\% of cases appear to function without any short-term battery storage
while the system cost rises by 4\% at most.
However, especially battery storage exhibits a large degree of freedom to build more.

%F contour plot at fixed epsilon

\begin{figure}
    \noindent\makebox[\textwidth]{
    \begin{subfigure}[t]{0.45\textwidth}
        \centering
        \caption{wind and solar}
        \label{fig:dependencies:ws}
        \includegraphics[width=\textwidth]{dependency/2D_surr-low-wind-solar.pdf}
    \end{subfigure}
    \begin{subfigure}[t]{0.45\textwidth}
        \centering
        \caption{offshore and onshore wind}
        \label{fig:dependencies:oo}
        \includegraphics[width=\textwidth]{dependency/2D_surr-low-offwind-onwind.pdf}
    \end{subfigure}
    \begin{subfigure}[t]{0.45\textwidth}
        \centering
        \caption{hydrogen and battery storage}
        \label{fig:dependencies:hb}
        \includegraphics[width=\textwidth]{dependency/2D_surr-low-H2-battery.pdf}
    \end{subfigure}
    }
    \caption{
      Space of near-optimal solutions by selected pairs of technologies under cost uncertainty.
    Just like in \cref{fig:fuzzycone}, the contour lines depict the overlap of the space of near-optimal alternatives across the parameter space.
    It can be thought of as the cross-section of the probabilistic near-optimal feasible space for a given $\varepsilon$
    in two technology dimensions and highlights that the extremes of two technologies from \cref{fig:fuzzycone} cannot be achieved simultaneously.
    }
    \label{fig:dependencies}
\end{figure}

The fuzzy cones from \cref{fig:fuzzycone} assume that the other technologies can be heavily optimised.
But as there are dependencies between the technologies, in \cref{fig:dependencies}
we furthermore present the probabilistic near-optimal space for three selected technology
pairs at fixed $\varepsilon=6\%$ for which we expect the most interesting trade-offs.
It addresses the question about which combinations of wind and solar capacity,
offshore and onshore turbines, and hydrogen and battery storage are likely to be achievable.

First, \cref{fig:dependencies:ws} addresses constraints between wind and solar.
The upper right boundary exists because building much of both wind and solar would bee too expensive.
The absence of solutions in the bottom left corner means that
building too little of any wind or solar does not suffice to produce enough electricity.
From the shape and contours, we see a high chance
that building 1000 GW of wind and 400 GW of solar is within 6\% of the cost-optimum.
On the other hand, building less than 200 GW of solar and 600 GW of wind is unlikely to yield a low-cost solution.
Minimising the capacity of both primal energy sources will shift capacity installations to
high-yield locations even if additional network expansion is necessary and boost
the preference for highly efficient storage technologies.
Overall, even considering combinations of wind and solar,
a wide space of low-cost options exists with moderate to high likelihood,
although the range of alternatives is shown to be more constrained.

\cref{fig:dependencies:oo} concerns the trade-off between onshore wind and offshore wind.
The most certain area is characterised by building more than 600 GW onshore wind,
and less than 250 GW offshore wind capacity.
However, there are some solutions with high substitutability between onshore and offshore wind,
shown in the upper left of the contour plot.
Compared to wind and solar, the range of near-optimal solutions is more constrained.
Finally, \cref{fig:dependencies:hb} underlines that some storage is definitely needed,
50 GW of each at least in any case. Highest likelihoods are attained when building 150 GW of each.

\begin{figure}
    \noindent\makebox[\textwidth]{
    \begin{subfigure}[t]{0.65\textwidth}
     \centering
     \caption{minimal onshore wind with 8\% system cost slack}
     \label{fig:nearviolin:onwind}
     \includegraphics[width=\textwidth, trim=0cm .3cm 3.3cm .63cm, clip]{violins/violin-capacities-high-prediction-min-onwind-0.08.pdf}   
    \end{subfigure}
    \begin{subfigure}[t]{0.65\textwidth}
     \centering
     \caption{minimal transmission expansion with 8\% system cost slack}
     \label{fig:nearviolin:transmission}
     \includegraphics[width=\textwidth, trim=0cm .3cm 3.3cm  .63cm, clip]{violins/violin-capacities-high-prediction-min-transmission-0.08.pdf}   
    \end{subfigure}
    }
    \caption{
      Distribution of total system cost, generation, storage, and transmission capacities
      for two near-optimal search directions with $\varepsilon=8\%$ system cost slack.
    }
    \label{fig:nearviolin}
\end{figure}

Neither of the aforementioned contour plots (\crefrange{fig:fuzzycone}{fig:dependencies}) expose
what changes the system layout experiences when reaching for the extremes in one technology.
Therefore, we show in \cref{fig:nearviolin} how the system-wide capacity distributions vary 
compared to the least-cost solutions (\cref{fig:violin}) for two exemplary alternative objectives.
For that, we chose minimising onshore wind capacity and transmission expansion
because they are often linked to public opposition.

\cref{fig:nearviolin:onwind} illustrates that reducing onshore wind capacity is
predominantly compensated by increased offshore wind generation but also added solar capacities.
The increased focus on offshore wind also leads to a tendency towards more hydrogen storage,
while transmission expansion levels are similarly distributed as for the least-cost solutions.
From \cref{fig:nearviolin:transmission} we can further extract that avoiding transmission expansion entails
more hydrogen storage that compensates balancing in space with balancing in time,
and more generation capacity overall, where resources are distributed to locations with
high demand but weaker capacity factors and more heavily curtailed.

\subsection{Critical Appraisal and Outlook}
%%\blindtext

Reconstruction and Disaggregation
- When using the surrogate model we do not get the full original model outputs.
- If we are interested in spatially explicit outcomes, we can either take one of the samples
or reconstruct selectively by adding the aggregate outcomes as additional constraints to the full model.

no sector-coupling

no path dependencies and multi-period investments

endogenous learning
- These cheap areas should then be targeted with policy (e.g. drive technology learning by subsidies)
- smoothly leads up to future work with endogenous learning in multi-horizon investment planning!
- embed technological learning in optimisation \cite{heuberger_power_2017} \cite{lopion_cost_2019}, remaining uncertainty of learning rate, but "learning by doing" included

further uncertain input parameters:
- rountrip efficiency of hydrogen storage

richer set of technologies?
- solar thermal, nuclear, CAES, separate hydrogen storage to fuel cell, biomass, geothermal, electrolysis, gas turbine


\usection{Discussion and Conclusion}{sec:conclusion}

% summary
In this work, we
systematically explore a space of alternatives beyond least-cost solutions
for society and politics to work with.
We show how blindly following cost-optimal results
underplays an immense degree of freedom in designing future renewable power systems.
To back our finding that there is no unique path to cost-efficiency,
we account for the inherent uncertainties regarding technology cost projections,
and draw conclusions about the range of options, boundary conditions and cost sensitivities:

\paragraph{Range of Options}
We find that there is a substantial range of options
within 8\% of the least-cost solution
regardless of how cost developments will unfold.
This holds across all technologies individually
and even when considering dependencies between
wind and solar, offshore and onshore wind, as well as hydrogen and battery storage.

\paragraph{Boundary Conditions}
We also carve out a few boundary conditions which
must be met to keep costs low and are not affected
by the prevailing cost uncertainty.
For a fully renewable power system,
either offshore or onshore wind capacities
in the order of 600 GW
along with some long-term storage technology and
transmission network reinforcement by more than 30\% appears essential.

\paragraph{Cost Sensitivities}
We identify onshore wind cost as the apparent main determinant of system cost,
though it can often be substituted with offshore wind for a small additional cost.
Moreover, the deployment of batteries is the most sensitive to its cost,
whereas required levels of transmission expansion are least affected by cost uncertainty. \\

% concluding remark
The robust investment flexibility in shaping a fully renewable power system
we reveal opens the floor to discussions about social trade-offs and navigating
around issues, such as public opposition towards wind turbines or transmission lines.
Recognising both the parametric and structural uncertainties of our modelling,
we realise that we--and others--can provide policy-makers with no more, but also no less,
than a framed canvas and the colors to paint.
% sketches
% Getting the artwork done is more important than getting it perfect!
% When commissioning studies, it is therefore important that policymakers
% rather ask for a range of plausible choices rather than solutions
% optimised to the last digit.


\usection{Experimental Procedures}{sec:methods}

\subsection{Power System Investment Planning}
\label{sec:model}
%\blindtext

\subsubsection{Power System Model}

%F map of 37 and 128 nodes and time series 4H/2H at particular location

\begin{SCfigure}
    \begin{tabular}{cc}
        \footnotesize low-fidelity: 37 nodes and 4-hourly & \footnotesize high-fidelity: 128 nodes and 2-hourly \\
        \includegraphics[width=0.33\textwidth]{map37.pdf} &
        \includegraphics[width=0.33\textwidth]{map128.pdf} \\
        \includegraphics[width=0.33\textwidth]{timeseries37.pdf} &
        \includegraphics[width=0.33\textwidth]{timeseries128.pdf} \\
    \end{tabular}
    \caption{Lorem ipsum dolor sit amet, consetetur sadipscing elitr, sed diam nonumy eirmod tempor invidunt ut labore et dolore magna aliquyam erat, sed diam voluptua.}
\end{SCfigure}

PyPSA-Eur \cite{pypsa} \cite{pypsaeur} \cite{hoersch_spatial_2017} \cite{snakemake}

only power system

100\% renewable system based on wind and solar

greenfield, exception: today's transmission grid, hydro, run-of-river


\subsubsection{Least-Cost Optimisation}

%objective
The objective is to minimise the total annual system costs comprising generation, transmission and storage infrastructure in a fully renewable system.

% constraints
The objective is subject to linear constraints that define limits on (i) the capacities of infrastructure from geographical and technical potentials, (ii) the availability of variable renewable energy sources for each location and point in time derived from reanalysis weather data, and (iii) linearised multi-period optimal power flow (LOPF) constraints including storage consistency equations.



\begin{equation}
    C = \min_x\{c^\top x \mid Ax\leq b\}
\end{equation}

\subsubsection{Near-Optimal Solutions}

Following \cite{nearoptimal}
epsilon constraint method from multi-objective optimisation \cite{mavrotas_effective_2009}

one technology

- new objective
- add constraint to limit cost increase

$x_s\subseteq x$ (for instance solar capacities)

\begin{align}
    \overline{x_s} = \max_{x_s}\{\:&x_s \mid Ax\leq b,\quad c^\top x\leq (1+\varepsilon)\cdot C \:\} \\
\end{align}

for generation (any wind, onshore wind, offshore wind, solar)
for transmission
for storage (hydrogen, battery)

two technologies

- add constraint to fix total capacity

$x_w\subseteq x$ (for instance wind capacities)

\begin{equation}
    \overline{x_w} = \max_{x_w}\{\:x_w \mid Ax\leq b,\quad c^\top x\leq (1+\varepsilon)\cdot C, \quad x_s = \underline{x_s} + \alpha \cdot (\overline{x_s}-\underline{x_s}) \:\}
\end{equation}

for pairs
(i) wind and solar,
(ii) offshore and onshore wind,
(iii) hydrogen and battery storage

more sophisticated than discrete steps to map the 2D crossection in \cite{pedersen_modeling_2020}

\subsection{Technology Cost Uncertainty}
\label{sec:uncertainty}
%\blindtext

Main challenge is quantifying the input uncertainties! \cite{moret_characterization_2017}

"the uncertainties inherent in the model structures and input parameters are at best underplayed and at worst ignored" \cite{yue_review_2018}

While quantifying cost uncertainties itself is uncertain,
it is more insightful than using best-guess values \cite{fraiture_robustness_2020}

Two main sources of uncertainty wrt technological learning \cite{trondle_trade-offs_2020}
- future deployment rates unknown
- learning rate unknown
- odels "highly sensitive to uncertainty in the learning rates [...] due to the exponential relationship" \cite{mattsson_learning_2019}
- \cite{yeh_review_2012}
- \cite{heuberger_power_2017}
- \cite{gritsevskyi_modeling_2000}
- \cite{schmidt_projecting_2019}
- \cite{schmidt_future_2017}

\subsubsection{Ranges}

\begin{SCtable}
    \begin{small}
        \begin{tabular}{cccc}
            \toprule
            Technology & Lower Annuity & Upper Annuity & Unit  \\ \midrule
            Onshore Wind & 73 & 109 & EUR/kW/a \\
            Offshore Wind & 178 & 245 & EUR/kW/a \\ % this includes connection cost!
            Solar & 36 & 53 & EUR/kW/a \\
            Battery & 30 & 125 & EUR/kW/a \\
            Hydrogen & 111 & 259 & EUR/kW/a \\ \bottomrule
        \end{tabular}
    \end{small}
    \caption{Technology cost uncertainty using optimistic and pessimistic assumptions from DEA.}
\end{SCtable}   

vital to choose widest plausible parameter range and not exclude plausible scenarios \cite{moret_characterization_2017,mccollum_energy_2020}

Danish Energy Agency source
- take optimistic and pessimistic values
- lower and higher bounds "shall be interpreted as representing probabilities corresponding to a 90\% confidence interval"
- for 2050, but supplemented with 2030 values if widen range
- continuously updated since 2016
- no ranges rooftop PV
- `./costcomparison.csv`

Comparison to others:
- \cite{trondle_trade-offs_2020} has almost always more conservative values than pessmistic DEA database; data mostly from ETIP
- JRC Energy Technology Reference Indicators is relatively old and conservative from 2014

Other uncertainty ranges:
- previous studies have considered "relatively narrow range[s] of uncertainties" \cite{Li2017}
- plus/minus 20\% \cite{moret_characterization_2017}
- mostly less than plus/minus 25\% \cite{pizarro-alonso_uncertainties_2019}
- plus/minus 50\% \cite{shirizadeh_how_2019}

\subsubsection{Distributions}

% literature

technology cost projection distributions
- uniform: \cite{moret_characterization_2017,moret_robust_2016,shirizadeh_how_2019,trondle_trade-offs_2020,pilpola_analyzing_2020,Li2017,Trutnevyte2013,lopion_cost_2019}
- normal: \cite{mavromatidis_uncertainty_2018}
- triangle: \cite{li_using_2020}
- "The variance [..] was set according to the maturity of technologies" \cite{li_using_2020}

always independently sampled (IID); easy but incorrect: e.g. offshore and onshore wind

"difficult and possibly misleading to associate a PDF to a parameter with unknown PDF" \cite{moret_robust_2016}

"To date, a general methodology for uncertainty characterization, assessing parameter uncertainty by type and degree [..] is missing." \cite{moret_robust_2016}

% maximum-entropy

"Following a maximum entropy approach, we model [...] uncertainty with uniform distributions over ranges taken from the literature" \cite{trondle_trade-offs_2020}
- %https://de.wikipedia.org/wiki/Maximum-Entropie-Methode
- %https://en.wikipedia.org/wiki/Principle_of_maximum_entropy
- from Bayesian statistics
- "assign an a-priori probability despite insufficient problem-specific information" (wiki)
- "the one that makes fewest assumptions about the true distribution of data" (wiki)
- % "entropy maximization with no testable information respects the universal constraint that the sum of the probabilities is one. Under this constraint, the maximum entropy probability distribution is the uniform distribution" (wiki)

\subsubsection{Sampling}

%https://chaospy.readthedocs.io/en/master/sampling/sequences.html
%https://en.wikipedia.org/wiki/Low-discrepancy_sequence
%https://en.wikipedia.org/wiki/Quasi-Monte_Carlo_method

using chaospy \cite{feinberg_chaospy_2015}

also called experimental design or collocation points

on optimal experimental design \cite{fajraoui_optimal_2017}

low-discrepancy series improvements to random sampling! avoid large gaps and clusters, deterministic sequences

can be used to efficiently sample from the parameter space.

traditional MC sampling with pseudo-random numbers has slow convergence % 1 / \sqrt{n}

strategies to find sufficiently accurate surrogate at low computational cost \cite{fajraoui_optimal_2017}

intendended to achieve "efficient coverage" \cite{usher_value_2015}

quasi Monte-Carlo (MC) sampling

using Halton sequence

alternatives are Latin hypercube sampling \cite{trondle_trade-offs_2020} and Method of Morris \cite{usher_value_2015,mavromatidis_uncertainty_2018}

Categories of sampling methods \cite{palar_multi-fidelity_2016}:
- structured: orthogonal polynomial roots, Newton-Cotes, sparse grid
- unstructured: random, LHS, low-discrepancy
- unstructured types typically used for uniform distributions

disadvantage of LHS \cite{fajraoui_optimal_2017}
- sampling size has to be set a priori, extensions difficult

Challenges
- adds more dimensions to uncertainty space, curse of dimensionality [12 in \cite{trondle_trade-offs_2020}, 36 in \cite{pilpola_analyzing_2020}, 5 in \cite{shirizadeh_how_2019}, 5 in this paper]

- \cite{trondle_trade-offs_2020} high fidelity: 10 samples, 400 nodes, 4-hourly; low fidelity: 150 samples, 25 nodes, 4-hourly; no DC power flow
- \cite{shirizadeh_how_2019} 315 cost samples

how many optimisation runs in this study
- 50000 low-fidelity optimisation runs
- 1000 high-fidelity runs

computational requirements:
- high-fidelity: 20 GB, 5 hours, 4 threads, Gurobi, average values, CPU time ~238 weeks or 4.56 years
- low-fidelity: 3 GB, 5 minutes, 1 thread, Gurobi, average values, CPU time ~25 weeeks or 0.48 years

\subsection{Surrogate Modelling}
\label{sec:surrogate}
%\blindtext

premise: outcome of original model cannot be obtained easily (e.g. computational constraints)

consider only simplified/aggregated outputs

only input/output behaviour is important (link to machine learning), each output can be predicted independently

synonyms for surrogate: emulators, approximation models, response surface methods, metamodels

same input/output behaviour as original model, but faster to compute

use cases: design space exploration, sensitivity analysis, what-if analyis

"reduce the number of deterministic evaluations while retaining accuracy" \cite{palar_multi-fidelity_2016}

used inputs: cost of offshore, onshore, solar, H2, battery
recorded outputs: system cost, wind, offwind, onwind, solar, H2, battery, transmission

\subsubsection{Polynomial Chaos Expansion}

literature
detailed general info and advise
\cite{gratiet_metamodel-based_2015} 
\cite{sudret_global_2008}
\cite{fajraoui_optimal_2017}
\cite{marelli_uqlab_nodate}


Use PCE to build surrogate models \cite{sudret_global_2008}

What is PCE?
- Hilbert space technique (is like a Euclidian space for polynomials, polynomials are so to speak the coordinates \cite{gratiet_metamodel-based_2015})
- polynomial expansion is a collection of polynomials
- needs finite variance!
- idea: "expand random variables as a linear combination of orthogonal basis functions weighted by deterministic coefficients." \cite{muhlpfordt_uncertainty_2020}
- "A polynomial chaos expansion for a random variable is what a classic Fourier series is for a periodic signal" \cite{muhlpfordt_uncertainty_2020}
- PCE is like PCA over an orthogonal vector space of polynomials
- "In practice the truncated PCE is used, meaning that only finitely many coefficients represent the random variable" \cite{muhlpfordt_uncertainty_2020}
- steps: find polynomial basis, calculate polynomial coefficients
- techniques for finding coefficients:
  - spectral projection: project response onto basis functions using inner products and orthogonal polynomials \cite{palar_multi-fidelity_2016}
  - point collocation: obtain coefficients by performing  regression \cite{palar_multi-fidelity_2016} regression problem in \cite{fajraoui_optimal_2017}
want to do non-intrusive (i.e. wrapping around model) point collocation (i.e. MC sampling) based on regression \cite{ng_multifidelity_2012}

one surrogate model for each $\varepsilon$ (5)  , min/max optimisation sense (2), objective (7) and output variable (8).

steps (following \cite{feinberg_chaospy_2015} and many others):
1 sample from parameter space (experimental design)
2 evaluate model at samples
3 select an expansion of orthogonal polynomials within the parameters space (Hilbertian basis of Hilbert space that contains the response \cite{sudret_global_2008})
4 do a regression for PCE coefficients
5 use the model approximation for statistial analysis

without truncation the polynomial chaos expansion is exact! \cite{fajraoui_optimal_2017}
standard truncation scheme \cite{gratiet_metamodel-based_2015,sudret_global_2008}

- typical degrees 3-5 \cite{gratiet_metamodel-based_2015}

how to create a polynomial chaos expansion
- three terms recurrence algorithm \cite{feinberg_chaospy_2015}

sparsity-of-effects principle:
- "system can be represented using a small number or statistically significant effects" \cite{berchier_multi-fidelity_nodate}
- in PCE: use only small number of polynomials from polynomial basis
- achieved by adding regularisation/penalty term to regression to favour sparse solutions (e.g. LARS regression)
- one can force sparse expansion (few non-zero coefficients) to partially counteract curse of dimensionality \cite{gratiet_metamodel-based_2015}
- these will use regularisation terms in the formulation of the regression problem
- but ordinary least-square regression is fine if number of uncertaint parameters is $\leq 10$ \cite{gratiet_metamodel-based_2015}

over-sampling ratio (OSR):
- ratio between number of samples and coefficients of polynomial basis (cardinality) \cite{palar_multi-fidelity_2016}
- cardinality increases exponentially with number of input parameters -> curse of dimensionality
- recommended values range between two and three \cite{hosder2007,palar_multi-fidelity_2016,fajraoui_optimal_2017,gratiet_metamodel-based_2015}
- too high OSR: very coarse approximation; too low OSR: risk of over-fitting \cite{palar_multi-fidelity_2016}

OSR considerations (M=5) 2-3
- for order 1, need 12-18 samples
- for order 2, need 42-63 samples
- for order 3, need 112-168 samples
- for order 4, need 252-378 samples
- for order 5, need 504-756 samples

% math equations

vector of random variables
\begin{equation}
    \bm{X} = \{X_1, \dots , X_M\}
\end{equation}

computational model
\begin{equation}
    Y = \cM ( \bm X )
\end{equation}

polynomial chaos expansion
\begin{equation}
    Y = \cM ( \bm X ) = \sum_{\ba \,\in\, \mathbb{N}^M} c_{\ba} \Psi_{\ba} (\bm X)
\end{equation}
where $\Psi_\ba$ is a collection of multivariate orthogonal polynomials that form a Hilbertian basis on $f_{\bm X}$
where $\ba = \{\alpha_1,\dots,\alpha_M\}$ is a multiindex and its elements
denotes the degree of $\Psi_\ba$ in each of the $M$ input variables $X_i$
$Y\in\mathbb{R}$ is scalar in the following, but we will look at 8 outputs
$c_\ba \in \mathbb{R}$ are the polynomial coefficients (or PCE coefficients)

orthogonality condition
\begin{equation}
    \langle \Psi_{\bm\alpha}, \Psi_{\bm\beta} \rangle = \delta_{\bm{\alpha\beta}}
\end{equation}
where $\delta_{\bm{\alpha\beta}}$ is the Kronecker delta which is $1$ if $\bm\alpha=\bm\beta$ and $0$ otherwise.

truncated polynomial chaos expansion
\begin{equation}
    \cM^{PC}(\bm X) = \sum_{\ba \,\in\, \cA^{M,p}} c_\ba \Psi_\ba(\bm X)
\end{equation}

standard truncation (select all polynomials in $M$ input variables where total degree is less than $p$), it's a set of indices
\begin{equation}
    \cA^{M,p} = \left\{\ba \in \mathbb{N}^M \,:\, \abs{\ba} \leq p\right\}
\end{equation}
where $\abs{\ba} = \sum_{i=1}^M \alpha_i$

cardinality of truncation
\begin{equation}
    P = \card \cA^{M,p} = \left(\begin{matrix}
        M+p \\
        p
    \end{matrix}\right) = \frac{(M+p)!}{M!p!}
\end{equation}
P: number of unknown coefficients, also cardinality
M: number of uncertain input parameters
p: order/degree of the polynomial

coefficients to be found
\begin{equation}
    \bm c = \left\{c_\ba \,:\, \ba \in \cA^{M,p}\right\}
\end{equation}

experimental design
inputs
\begin{equation}
    \mathcal{X} = \set{ \bm x^{(1)},\dots,\bm x^{(N)} }
\end{equation}
outputs
\begin{equation}
    \mathcal{Y} = \set{ \cM\left(\bm x^{(1)}\right),\dots,\cM\left(\bm x^{(N)}\right) }
\end{equation}

regression: minimise least-square residual of polynomial approximation over all samples
\begin{equation}
    \hat{\bm{c}} = \argmin_{\bm{c}_\ba \,\in\, \mathbb{R}^P} \frac{1}{N} \sum_{i=1}^N \left(
        \cM \left(\bm x^{(i)}\right) - \sum_{\ba \,\in\, \cA^{M,p}} c_\ba \Psi_\ba\left(\bm x^{(i)}\right)
    \right)^2
\end{equation}

finished surrogate model
\begin{equation}
    Y = \cM(\bm X) \approx \cM^{PC}(\bm X) = \sum_{\ba \,\in\, \cA} \hat{c}_\ba \Psi_\ba (\bm X)
\end{equation}

\subsubsection{Multifidelity Approach}

\cite{palar_multi-fidelity_2016}
\cite{ng_multifidelity_2012}
\cite{berchier_multi-fidelity_nodate}

fuse high coverage of parameter space from many low-fidelity samples with
high modelling detail from fewer high-fidelity samples

use a simplified model to sweep the parameter space (many samples, fewer details)
and supplement it with information from a more complex model (fewer samples, more detail)

following (Jiant, 2019)

only high-fidelity for surrogate model is time-consuming,
but only low-fidelity may result in distorted/inaccurate surrogate models

"If LF model is much cheaper to evaluate than the HF model, then significant
computational savings will be obtained" \cite{ng_multifidelity_2012}

integrate both with multi-fidelity approach: correct LF model such that LF output matches HF values

build an own surrogate model for the correction function (i.e. "correction expansion") with identical polynomial basis

concept is "correcting lower order effect of LF expansion with correction function, higher order effects are predicted by LF expansion" \cite{palar_multi-fidelity_2016}

rule: "polynomial term in the correction expansion has to e a subset of the LF expansion" \cite{palar_multi-fidelity_2016}
-> has been checekd, means high-fidelity polynomial order must be lower 

HF sample is subset of LF sample (this is a nice feature of low-discrepancy series)

"in cases with high correlation between LF an HF ($R^2 \geq 0.9$) the MF will very likely increase the approximation accuracy relative to single HF" \cite{palar_multi-fidelity_2016}

method:
- build surrogate A based on low-fidelity samples
- build corrective surrogate B for the difference between high-fidelity samples and predictions of A at those sample points
- add the correction B to the low-fidelity surrogate A to obtain multi-fidelity surrogate C which approximates the high-fidelity model

Additive scaling approach (argue why additive is used not multiplicative)
- Correct for the difference between HF samples and LF surrogate at HF sample points
% \begin{equation}
%     \tilde{f}_{MF}(\xi) = \tilde{f}_{LF} + \tilde{\alpha}(\xi)
% \end{equation}
% where $\tilde{\alpha}(\xi)$ is the additive scaling function. The sampled scaling factors are calculated with
% \begin{equation}
% \alpha(\xi) = f_{HF}(\xi) - \tilde{f}_{LF}(\xi)
% \end{equation}
% Construct scaling function $\tilde{\alpha}(\xi)$ from scaling factors $\alpha(\xi)$ via some regression fit.

error function
\begin{equation}
    \mathcal{C}(\bm X) = \cM_h(\bm X) - \cM_\ell(\bm X)
\end{equation}

samples
\begin{align}
    \mathcal{X}_h &= \set{ \bm{x}^{(1)}, \dots, \bm{x}^{(N_h)}} \\
    \mathcal{X}_l &= \set{ \bm{x}^{(1)}, \dots, \bm{x}^{(N_h)}, \dots, \bm{x}^{(N_l)}}
\end{align}
where $N_h < N_\ell$ and when using deterministic low-dicrepancy series
consequently $\mathcal{X}_h \subset \mathcal{X}_\ell$.

Due to the smaller number of samples for the high fidelity model (and therefore error between high and low fidelity), it is also
common that $p_h < p_l$ and consequently $\cA_h \subset \cA_l$

multifidelity model (simple)
\begin{equation}
    \cM_h^{PC}(\bm X) = \cM_\ell^{PC}(\bm X) + \mathcal{C}^{PC}(\bm X)
\end{equation}

multifidelity model (more detailed)
\begin{equation}
    \cM_h^{PC} (\bm X) = \sum_{\ba\,\in\,\cA_\ell^{M,p_\ell} \,\cap\, \cA_c^{M,p_c}}
    \left(
     c_{\ell,\ba} + c_{c,\ba}
    \right) \Psi_\ba(\bm X) + 
    \sum_{\ba\,\in\,\cA_\ell^{M,p_\ell} \,\setminus\, \cA_c^{M,p_c}}
    c_{\ell,\ba} \Psi_\ba(\bm X)
\end{equation}

\subsection{Evaluation Methods}
\label{sec:evaluationmethods}

\subsubsection{Sensitivity Indices}

Sobol: decomposition of output variance and attribution to random input variables

global method!

Sobol first-order second-order total

first-order Sobol:
- also main-effect
- share of output variance due to variations in one input parameter alone (averaged over variations in other input parameters)

total Sobol indices:
- contribution to output variance iincluding all interactions with other input variables
- can be greater than 100\% if not purely additive

Sensitivity indices:
- separate influential from non-influential parameters
- computational cost of sensitivity indices reduces to that of estimating PCE coefficients, 2-3 orders of magnitude faster than traditional MC evaluation \cite{sudret_global_2008}
- The polynomials can be used for analytic calculations (like Sobol indices) as post-processing and thereby have added value compared to pure machine learning approaches

used in \cite{trondle_trade-offs_2020,mavromatidis_uncertainty_2018}

\subsubsection{Validation and Error Measures}

error estimation, measures of accuracy

using cross-validation techniques 
- training set of known data
- validation/test set of unknown data to surrogate \cite{gratiet_metamodel-based_2015}
- only one split, not multiple random splits of test/train sets

error measures:
- R2: quality of the regression "proportion of the variance in the dependent variable that is predictable from the independent variable" (wiki)
- MAE: to see absolute deviations (in GW), alternative to relative error (esp if prediction is 0 relative errors are super high)
- MAPE: relative deviations

- based on low-fidelity models only (too few samples for error analysis for high-fidelity model)
- out of 500 cost samples, 100 have not been used for building/training the surrogate model. This is our test set.

- benchmark: model error below 5\%, but not specific \cite{trondle_trade-offs_2020}


\section*{Acknowledgement}

F.N. and T.B. gratefully acknowledge funding from the Helmholtz Association
under grant no. VH-NG-1352. The responsibility for the contents lies with the
authors.

\section*{License}

\href{http://creativecommons.org/licenses/by/4.0/}{Creative Commons Attribution 4.0
International License (CC-BY-4.0)}

\section*{CRediT Author Statement}

\textbf{Fabian Neumann:} Conceptualization, Methodology, Investigation, Software, Validation, Formal analysis, Visualization, Writing -- Original Draft, Writing -- Review \& Editing
\textbf{Tom Brown:} Conceptualization, Writing -- Review \& Editing, Supervision, Project administration, Funding acquisition

\section*{Data and Code Availability}

The code to reproduce the experiments as well as results dat including selected
networks and all graphics is available at
\href{https://github.com/fneum/broad-ranges}{github.com/fneum/broad-ranges} and
archived at
\href{https://doi.org/10.5281/zenodo.6641551}{doi:10.5281/zenodo.6641551}. We
also refer to the documentation of PyPSA
(\href{https://pypsa.readthedocs.io}{pypsa.readthedocs.io}) and PyPSA-Eur
(\href{https://pypsa-eur.readthedocs.io}{pypsa-eur.readthedocs.io}).

% tidy with https://flamingtempura.github.io/bibtex-tidy/
\addcontentsline{toc}{section}{References}
\renewcommand{\ttdefault}{\sfdefault}
\bibliography{library}

\newpage
\tableofcontents

% \begin{appendix}
% 	\input{sections/appendix.tex}
% \end{appendix}

\end{document}