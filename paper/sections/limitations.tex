The need to solve models for many cost projections and near-optimal search
directions in reasonable time means that compromises had to be made in other
modelling dimensions. For instance, the analysis would profit from a richer set
of technologies and further uncertain input parameters, including efficiencies
of fuel cells and electrolysis or the consideration of concentrating solar
power, geothermal energy, biomass, and nuclear to name just a few. All of these
may also influence the near-optimal range of options for the technologies we
considered. But as the number of considered technologies and parameters rises,
so does the computational burden. Given the already considerable computational
efforts involved in procuring our results, considering the full breadth of
technologies and uncertainties would not have been feasible with the
computational resources available. Moreover, further limitations apply to the
scope of the analysis which is limited to the electricity sector and does not
consider rising electricity demand as also other energy sectors are electrified.
Like a broader set of technologies, leveraging additional measures to integrate
renewables through tighter cross-sectoral coupling and demand-side flexibilities
would also yield different results about the technology choices for near-optimal
energy system designs. Therefore, accounting for interactions across sectors at
high resolution in a similar future study is desirable. Additionally, we assess
no path dependencies via multi-period investments and endogenous learning, but
optimise for an emission reduction in a particular target system based on
annualised costs. For computational reasons, we disregard interannual variations
of weather data by basing the analysis just on a single weather year for
computational reasons, as well as uncertainties about future demand predictions
and electrification rates. Finally, aspects such as reserves, system adequacy
and inertia have not been considered.
