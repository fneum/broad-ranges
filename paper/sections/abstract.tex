Achieving ambitious CO$_2$ emission reduction targets requires energy system
planning to accommodate societal preferences, such as transmission
reinforcements or onshore wind parks, and acknowledge uncertainties in
technology cost projections among many other uncertainties. Current models often
solely minimise costs using a single set of cost projections. Here, we apply
multi-objective optimisation techniques in a fully renewable European
electricity system to explore trade-offs between system costs and technology
deployment for electricity generation, storage, and transport. We identify
ranges of cost-efficient capacity expansion plans incorporating future
technology cost uncertainties. For example, we find that some grid
reinforcement, long-term storage, and large wind capacities are important to
keep costs within 8\% of least-cost solutions. Near the cost-optimum a
technologically diverse spectrum of options exist, allowing policymakers to make
trade-offs regarding unpopular infrastructure. Our analysis comprises 50,000+
optimisation runs, managed efficiently through multi-fidelity surrogate
modelling techniques using sparse polynomial chaos expansions and
low-discrepancy sampling.
