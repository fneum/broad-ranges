% Nature Energy 150 words

To achieve ambitious greenhouse gas emission reduction targets in time, the
planning of future energy systems needs to accommodate societal preferences,
e.g.~low levels of acceptance for transmission expansion or onshore wind
turbines, and must also acknowledge the inherent uncertainties of technology
cost projections. To date, however, many capacity expansion models lean heavily
towards only minimising system cost and only studying few cost projections.
Here, we address both criticisms in unison. While taking account of technology
cost uncertainties, we apply methods from multi-objective optimisation to
explore trade-offs in a fully renewable European electricity system between
increasing system cost and extremising the use of individual technologies for
generating, storing and transmitting electricity to build robust insights about
what actions are viable within given cost ranges. We identify boundary
conditions that must be met for cost-efficiency regardless of how cost
developments will unfold; for instance, that some grid reinforcement and
long-term storage alongside a significant amount of wind capacity appear
essential. But, foremost, we reveal that near the cost-optimum a broad spectrum
of regionally and technologically diverse options exists in any case, which
allows policymakers to navigate around public acceptance issues. The analysis
requires to manage many computationally demanding scenario runs efficiently, for
which we leverage multi-fidelity surrogate modelling techniques using sparse
polynomial chaos expansions and low-discrepancy sampling.
