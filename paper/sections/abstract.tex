To achieve ambitious CO$_2$ emission reduction targets quickly, the planning of
energy systems must accommodate societal preferences, e.g.~regarding
transmission reinforcements or onshore wind parks, and must also acknowledge
uncertainties of technology cost projections among many other uncertainties. To
date, however, many models lean towards only minimising system cost and only
using a single set of cost projections. Here, we address both criticisms in
unison. While taking account of cost uncertainties, we apply multi-objective
optimisation techniques to explore trade-offs in a fully renewable European
electricity system between rising system cost and the deployment of individual
technologies for generating, storing and transporting electricity. We identify
ranges of capacity expansion plans that are cost-efficient and incorporate
uncertainty about future technology cost developments; for instance, we find
that some grid reinforcement and long-term storage alongside large wind
capacities are important to keep costs within 8\% of the least-cost solutions.
We reveal that near the cost-optimum a broad spectrum of technologically diverse
options exist, which allows policymakers to make trade-offs regarding unpopular
infrastructure. The analysis requires to manage many computationally demanding
scenario runs efficiently, for which we leverage multi-fidelity surrogate
modelling techniques using sparse polynomial chaos expansions and
low-discrepancy sampling.
