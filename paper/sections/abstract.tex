% Nature Energy 150 words

To achieve ambitious CO$_2$ emission reduction targets in time, the planning of
energy systems must accommodate societal preferences, e.g.~regarding
transmission reinforcements or onshore wind parks, and must also acknowledge
uncertainties of technology cost projections. To date, however, many models lean
towards only minimising system cost and only studying few cost projections.
Here, we address both criticisms in unison. While taking account of cost
uncertainties, we apply multi-objective optimisation techniques to explore
trade-offs in a fully renewable European electricity system between rising
system cost and the deployment of individual technologies for generating,
storing and transporting electricity. We identify boundary conditions that must
be met for cost-efficiency regardless of how cost developments will unfold; for
instance, that some grid reinforcement and long-term storage alongside large
wind capacities appear essential. But, foremost, we reveal that near the
cost-optimum a broad spectrum of technologically diverse options exist, which
allows policymakers to circumvent public opposition.

% The analysis requires to manage many computationally demanding scenario runs
% efficiently, for which we leverage multi-fidelity surrogate modelling
% techniques using sparse polynomial chaos expansions and low-discrepancy
% sampling.
