% summary

In this work, we systematically explore a space of alternatives beyond
least-cost solutions for society and politics to work with subject to uncertain
technology cost projections. We show how narrowly following cost-optimal results
underplays an immense degree of freedom in designing future renewable power
systems. To make our finding that there is no unique path to cost-efficiency
more robust, we account for technology cost uncertainties as one example of the
many unknowns faced in the energy transition, and draw the following
conclusions:

\paragraph{Wide Range of Trade-Offs}
We find that there is a substantial range of options within 8\% of the
least-cost fully renewable electricity system regardless of how cost
developments will unfold. This holds across all technologies individually and
even when considering dependencies between wind and solar, offshore and onshore
wind, as well as hydrogen and battery storage as examples of flexibility
options.

\paragraph{Solutions to Avoid}
We also carve out parts of the solution space which are unlikely to keep costs
within given cost ranges given the considered range of technology cost futures.
For a fully renewable electricity system, either offshore or onshore wind
capacities of the order of 600 GW along with some long-term storage technology
and transmission network reinforcement appears essential in the scenarios we
analyse. Less wind capacity leads to high cost solutions in our model.

\paragraph{Key Technology Cost Sensitivities}
We identify onshore wind cost as the apparent main determinant of system cost,
though it can often be substituted with offshore wind for a small additional
cost. This aligns with the finding that the near-optimal feasible space is flat.
Moreover, the deployment of batteries is the most sensitive to its cost, whereas
required levels of transmission expansion are least affected since transmission
cost were not considered to be uncertain.

\paragraph{Benefits of Combining MGA and Global Sensitivity Analysis}
The combination of modelling-to-generate-alternatives (MGA) to explore the
near-optimal solution space and global sensitivity analysis to account for an
uncertain input parameter space unifies two approaches to uncertainty
quantification. The presented methodology is helpful to show that near-optimal
insights are robust to some uncertainty (in our case technology cost). Likewise,
it can show whether some parts of the near-optimal solution space are more or
less affected by uncertainty.

% concluding remark

The robust finding of our study is that there is consistent investment
flexibility in shaping fully renewable power systems, even without availing of
the myriad flexibility options offered through sector-coupling. This opens the
floor to discussions about social trade-offs and navigating around issues, such
as public opposition towards wind turbines or transmission lines. Rather than
modellers making normative choices about how the energy system should be
optimised, by applying multi-fidelity surrogate modelling techniques and the
modelling-to-generate-alternatives methodology we offer a methodology to present
a wide spectrum of options and trade-offs that are feasible and within a
reasonable cost range, to help society decide how to shape the future of the
energy system.
