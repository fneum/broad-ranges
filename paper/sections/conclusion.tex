% summary
In this work, we systematically explore a space of alternatives beyond
least-cost solutions for society and politics to work with. We show how narrowly
following cost-optimal results underplays an immense degree of freedom in
designing future renewable power systems. To back our finding that there is no
unique path to cost-efficiency, we account for the inherent uncertainties
regarding technology cost projections, and draw robust conclusions about the
range of options, boundary conditions and cost sensitivities:

\paragraph{Wide Range of Trade-Offs}
We find that there is a substantial range of options
within 8\% of the least-cost solution
regardless of how cost developments will unfold.
This holds across all technologies individually
and even when considering dependencies between
wind and solar, offshore and onshore wind, as well as hydrogen and battery storage.

\paragraph{Must-Avoid Boundary Conditions}
We also carve out a few boundary conditions which
must be met to keep costs low and are not affected
by the prevailing cost uncertainty.
For a fully renewable power system,
either offshore or onshore wind capacities
in the order of 600 GW
along with some long-term storage technology and
transmission network reinforcement by more than 30\% appears essential.

% \paragraph{Robustness to Cost and Near-Optimal Perturbations}

\paragraph{Technology Cost Sensitivities}
We identify onshore wind cost as the apparent main determinant of system cost,
though it can often be substituted with offshore wind for a small additional cost.
Moreover, the deployment of batteries is the most sensitive to its cost,
whereas required levels of transmission expansion are least affected by cost uncertainty. \\

% concluding remark
The robust investment flexibility in shaping a fully renewable power system we
reveal opens the floor to discussions about social trade-offs and navigating
around issues, such as public opposition towards wind turbines or transmission
lines. Rather than modellers making normative choices about how the energy
system should be optimised, we offer methods that present a wide spectrum of
options and trade-offs that are feasible and within a reasonable cost range, to
help society decide how to shape the future of the energy system.

% ------------------- not used ----------------------------

% Recognising both the parametric and structural uncertainties of our modelling,
% we realise that we--and others--can provide policy-makers with no more, but also no less,
% than a framed canvas and the colors to paint.
% Getting the artwork done is more important than getting it perfect!
% When commissioning studies, it is therefore important that policymakers
% rather ask for a range of plausible choices rather than solutions
% optimised to the last digit.
