\usubsection{System Cost and Capacity Distribution of Least-Cost Solutions}{sec:sub:dist}

\begin{figure}
    \centering
    \includegraphics[width=.95\textwidth]{violins/violin-capacities-high-prediction.pdf}
    \caption{
      Distribution of system cost, generation, storage, and transmission
      in least-cost solutions. The dashed line in the
      transmission line chart indicates today's existing transmission
      capacities for comparison. Violin range is limited to the range of observed data.
      Boxplots show median, interquartile range, and upper/lower quartile $\pm 1.5$ times the interquartile range.
    }
    \label{fig:violin}
\end{figure}

We explore the impacts of cost uncertainty in a spatially and temporally
resolved model of the European power system with fully renewable generation and
zero direct carbon dioxide emissions. Based on sampling the uncertainty of cost
inputs given by the DEA,\cite{DEA} the total annual power system costs vary
between 160 and 220 bn\euro/a, as displayed in \cref{fig:violin}. This means the
most pessimistic cost projections entail about 40\% higher cost than the most
optimistic projections. All least-cost solutions build more than 350 GW solar
and 600 GW wind, but none more than 1100 GW of wind or more than 950 GW of
solar. While wind capacities tend to cluster towards higher values, solar
capacities tend towards lower values. We observe that least-cost solutions
prefer onshore over offshore wind, yet onshore wind features the highest
uncertainty range alongside battery storage. The cost optimum gravitates towards
hydrogen storage rather than battery storage unless battery storage becomes very
cheap. In the uncertainty space sampled, there are no least-cost solutions
without the long-duration storage provided by hydrogen, only some without
battery storage. Transmission network expansion is least affected by cost
uncertainty and consistently doubled compared to today's capacities to achieve a
fully renewable electricity system.

Although these results outline the extent to which cost uncertainty affects
cost-optimal designs, the insights from the observed ranges are limited because
there is considerable flexibility beyond the least-cost solutions and acknowledge
structural modelling uncertainties, such as social constraints to the expansion
of grids and wind turbines. Moreover, the pure distribution of outputs does not
yet convey information about how sensitive results are to particular cost
assumptions. But knowing the technologies for which lowering overnight costs has
a significant impact is important to promote technological learning in that
direction.

\begin{figure}
    \begin{subfigure}[t]{0.32\textwidth}
        \caption{onshore wind}
        \includegraphics[width=\textwidth]{1D/1D-onwind-onwind-high.pdf}
    \end{subfigure}
    \begin{subfigure}[t]{0.32\textwidth}
        \caption{offshore wind}
        \includegraphics[width=\textwidth]{1D/1D-offwind-offwind-high.pdf}
    \end{subfigure}
    \begin{subfigure}[t]{0.32\textwidth}
        \caption{solar}
        \includegraphics[width=\textwidth]{1D/1D-solar-solar-high.pdf}
    \end{subfigure} \\
    \begin{subfigure}[t]{0.32\textwidth}
        \caption{battery storage}
        \includegraphics[width=\textwidth]{1D/1D-battery-battery-high.pdf}
    \end{subfigure}
    \begin{subfigure}[t]{0.32\textwidth}
        \caption{hydrogen storage}
        \includegraphics[width=\textwidth]{1D/1D-H2-H2-high.pdf}
    \end{subfigure}
    \begin{subfigure}[t]{0.32\textwidth}
        \caption{transmission}
        \includegraphics[width=\textwidth]{1D/1D-transmission-H2-high.pdf}
    \end{subfigure}
    \vspace{-0.3cm}
    \caption{
      Sensitivity of capacities towards their own technology cost.
      The median (Q50) alongside the 5\%, 25\%, 75\%, and 95\% quantiles (Q5--Q95) display
      the sensitivity subject to the uncertainty induced by other cost parameters.
      Dots represent samples of the high-fidelity model runs.
    }
    \label{fig:sensitivity}
\end{figure}

\begin{figure}
    \begin{subfigure}[t]{0.45\textwidth}
        \caption{first-order Sobol indices [\%]}
        \label{fig:sobol:first}
        \includegraphics[width=\textwidth]{sobol/sobol-m-high.pdf}
    \end{subfigure}
    \begin{subfigure}[t]{0.54\textwidth}
        \caption{total Sobol indices [\%]}
        \label{fig:sobol:total}
        \includegraphics[width=\textwidth]{sobol/sobol-t-high-bar.pdf}
    \end{subfigure}
    \vspace{-0.3cm}
    \caption[First-order and total Sobol indices]{
      Sobol indices. These sensitivity indices attribute output variance to random input variables
      and reveal which inputs the outputs are most sensitive to. The first-order Sobol indices
      quantify the share of output variance due to variations in one input parameter alone.
      The total Sobol indices further include interactions with other input variables.
      Total Sobol indices can be greater than 100\% if the contributions are not purely additive.
    }
    \label{fig:sobol}
\end{figure}

\usubsection{Local Parameter Sweeps and Global Sensitivity Indices}{sec:sub:sobol}

Therefore, \cref{fig:sensitivity} expands the previous view by additionally
showing how the cost of a technology influences its deployment while displaying
the remaining uncertainty induced by other cost parameters. The overall tendency
is easily explained: the cheaper a technology becomes, the more it is built.
However, changes of slope and effects on the uncertainty range as one cost
parameter is swept are insightful still. For instance, \cref{fig:sensitivity}
reveals that battery storage becomes significantly more attractive economically
once its cost falls below 750 EUR/kW (including 6-hour energy capacity at full
power output), while hydrogen storage (including electrolysis, fuel cell and
underground storage with an energy-to-power ratio of 168 hours) features a
steady slope. A low cost of onshore wind makes building much onshore wind
capacity attractive with low uncertainty, whereas if onshore wind costs are high
how much is built greatly depends on other cost parameters. The opposite
behaviour is observed for offshore wind and solar. The cost of hydrogen storage
mostly causes the limited uncertainty about cost-optimal levels of grid
expansion. As the cost of hydrogen storage falls, less grid reinforcement is
chosen.

But since the presented sensitivity of capacities towards their own cost only
exhibit a fraction of all sensitivities, we further apply a more systematic
variance-based global sensitivity analysis, which have been applied in energy
systems analysis e.g.~in Tröndle~et~al.\cite{trondle_trade-offs_2020} and
Mavromatidis~et~al.\cite{mavromatidis_uncertainty_2018} Sensitivity indices, or
Sobol indices, attribute the observed output variance to each input.
\cite{sudret_global_2008} For our application, the Sobol indices can, for
instance, tell us which technology cost contributes the most to total system
cost or how much of a specific technology will be built. The first-order Sobol
indices describe the share of output variance due to variations in one input
alone averaged over variations in the other inputs. Total Sobol indices also
consider higher-order interactions, which are greater than 100\% if the
relations are not purely additive or independent.

The first-order and total Sobol indices for least-cost solutions in
\cref{fig:sobol} show that the total system cost is largely determined by how
expensive it is to build onshore wind capacity, followed by the cost of hydrogen
storage. The amount of wind in the system is almost exclusively governed by the
cost of onshore and offshore wind parks. Other carriers yield a more varied
picture. The cost-optimal solar capacities additionally depend on onshore wind
and battery costs. The amount of hydrogen storage is influenced by battery and
hydrogen storage cost alike. Although there are noticeable higher-order effects,
which are most extensive for transmission, the first-order effects dominate.
Strikingly, the volume of transmission network expansion strongly depends on the
cost of hydrogen storage. This can be explained because they both compete to
balance out the large weather systems crossing the continent which cause in
particular wind variations. Hydrogen storage can balance the multi-week transit
of weather systems in time, whereas transmission networks can smooth them in space.
While hydrogen storage typically balances multi-week variations in time,
continent-spanning transmission networks exploit the circumstance that as
weather systems traverse the continent, it is likely always to be windy
somewhere in Europe.

\usubsection{Fuzzy Near-Optimal Corridors with Increasing Cost Slack}{sec:sub:fuzzy}

So far, we quantified the output uncertainty and analysed the sensitivity
towards inputs at least-cost solutions only. However, it has been previously
shown that even for a single cost parameter set a wide array of technologically
diverse but similarly costly solutions
exists.\cite{nearoptimal,pedersen_modeling_2020,lombardi_policy_2020,pickeringDiversityOptions2022}
We now examine how technology cost uncertainty affects the shape of the space of
near-optimal alternatives within 8\% of the least-cost solution. We do not
extend beyond this value because the rate of change of most Pareto fronts has
considerably reduced at that point, while we acknowledge that higher cost
penalties may still be acceptable.

By identifying feasible alternatives common to all, few or no cost samples, in
\cref{fig:fuzzycone} we outline low-cost solutions common to most parameter sets
(e.g. above 90\% contour) as well as system layouts that do not meet low-cost
criteria for nearly no technology cost samples for varying $\epsilon$. For each
technology and cost sample, the minimum and maximum capacities obtained for
increasing cost penalties $\epsilon$ form a cone of an upper and a lower Pareto
front, starting from a common least-cost solution. These Pareto fronts delineate
boundaries beyond which neither reducing system cost nor extremising the
capacity of a technology can be improved without depressing the other. By
arguments of convexity, the capacity ranges contained by the cone can be
near-optimal and feasible, given a degree of freedom in the other technologies.
From optimisation theory, we also know that the cones widen up for increased
slacks. As we consider technology cost uncertainty, the cone will look slightly
different for each sample causing the fuzziness of the boundaries. The contour
lines represent the frequency with which a solution is inside the near-optimal
cone over the whole parameter space. This is calculated from the overlap of many
cones, each representing a different set of cost assumptions. The wider the
displayed contour lines are apart, the more uncertainty exists about the
borders. The closer contour lines are together, the more specific the limits are
despite the cost uncertainty. The height of the quantiles quantifies flexibility
for a given level of certainty and slack; the angle presents information about
the sensitivity towards cost slack.

\begin{figure}
    \vspace{-2cm}
    \noindent\makebox[\textwidth]{
    \begin{subfigure}[t]{0.45\textwidth}
        \centering
        % \caption{any wind}
        \includegraphics[width=\textwidth]{neardensity/surr-high-wind.pdf}
    \end{subfigure}
    \begin{subfigure}[t]{0.45\textwidth}
        \centering
        % \caption{onshore wind}
        \includegraphics[width=\textwidth]{neardensity/surr-high-onwind.pdf}
    \end{subfigure}
    \begin{subfigure}[t]{0.45\textwidth}
        \centering
        % \caption{offshore wind}
        \includegraphics[width=\textwidth]{neardensity/surr-high-offwind.pdf}
    \end{subfigure}
    }
    \noindent\makebox[\textwidth]{
    \begin{subfigure}[t]{0.45\textwidth}
        \centering
        % \caption{solar}
        \includegraphics[width=\textwidth]{neardensity/surr-high-solar.pdf}
    \end{subfigure}
    \begin{subfigure}[t]{0.45\textwidth}
        \centering
        % \caption{transmission network}
        \includegraphics[width=\textwidth]{neardensity/surr-high-transmission.pdf}
    \end{subfigure}
    }
    \noindent\makebox[\textwidth]{
        \begin{subfigure}[t]{0.45\textwidth}
            \centering
            % \caption{hydrogen storage}
            \includegraphics[width=\textwidth]{neardensity/surr-low-H2.pdf}
        \end{subfigure}
        \begin{subfigure}[t]{0.45\textwidth}
            \centering
        % \caption{battery storage}
        \includegraphics[width=\textwidth]{neardensity/surr-low-battery.pdf}
    \end{subfigure}
    }
    \caption{
    Space of near-optimal solutions by technology under cost uncertainty.
    For each technology and cost sample,
    the minimum and maximum capacities obtained for increasing cost penalties
    $\epsilon$ form a cone of an upper and a lower Pareto front, starting from a common least-cost solution.
    By arguments of convexity, the capacity ranges contained by the cone can be near-optimal and feasible, given a degree of freedom in the other technologies.
    From optimisation theory, we know that the cones widen up for increased slacks.
    As we consider technology cost uncertainty, the cone will look slightly different for each sample.
    The contour lines represent the frequency a solution is inside the near-optimal cone over the whole parameter space.
    This is calculated from the overlap of many cones, each representing a set of cost assumptions.
    Due to discrete sampling points in the $\epsilon$-dimension, the plots further apply quadratic interpolation and a Gaussian filter for smoothing.
    }
    \label{fig:fuzzycone}
\end{figure}

From the fuzzy upper and lower Pareto fronts in \cref{fig:fuzzycone} it can be
seen that for our scenarios, building 900 GW of wind capacity is highly likely
possible within 3\% of the optimum, and that conversely building less than 600
GW has a low chance of being near the cost optimum with our model setup. Only a
few solutions can forego onshore wind entirely and remain within 8\% of the
cost-optimum, whereas it appears to be likely possible to build a system without
offshore wind at a cost penalty of 4\% at most. On the other hand, more offshore
wind generation seems equally possible. Unlike for onshore wind, where it is more
uncertain how little can be built, uncertainty regarding offshore wind
deployment exists about how much can be built so that costs remain within a
pre-specified range. For solar, the range of options within 8\% of the cost
optimum at 90\% certainty is very wide. Anything between 100 GW and 1000 GW
appears feasible as long as other substituting technologies are built and
suitably sited. In comparison to onshore wind, the uncertainty about minimal
solar requirements is smaller.

The level of required transmission expansion is least affected by the cost
uncertainty. To remain within the pre-defined $\epsilon=8\%$ it is just as
likely feasible to plan for moderate grid reinforcement by 30\% as is initiating
extensive remodelling of the grid by tripling the transmission volume compared
to what is currently in operation. One reason for this is perhaps that cost
uncertainty on building new transmission lines was not included as it is a quite
mature technology. These results indicate that in any of the cases considered
some transmission reinforcement to balance renewable variations across the
continent appears to be essential. Hydrogen storage, symbolising medium- to
long-term storage, also is a vital technology in many cases. In a model with
increased cross-sectoral integration, this role could also likely be taken over
by thermal storage or other power-to-X conversion processes. Some short- to
medium-term balancing needs might also be covered by demand-side management. At
$\epsilon=8\%$, only 25\% of cost samples require no long-term storage; namely
when battery costs are exceptionally low. Overall, 90\% of cases appear to
function without any short-term battery storage while the system cost rises by
4\% at most. However, especially battery storage exhibits a large degree of
freedom to build more given the high cost uncertainty reported in the DEA
technology database \cite{DEA}.

\usubsection{Probabilistic Near-Optimal Feasible Space in Two Technology Dimensions}{sec:sub:two-dim}

\begin{figure}
    \noindent\makebox[\textwidth]{
    \begin{subfigure}[t]{0.45\textwidth}
        \centering
        \caption{wind and solar}
        \label{fig:dependencies:ws}
        \includegraphics[width=\textwidth]{dependency/2D_surr-low-wind-solar.pdf}
    \end{subfigure}
    \begin{subfigure}[t]{0.45\textwidth}
        \centering
        \caption{offshore and onshore wind}
        \label{fig:dependencies:oo}
        \includegraphics[width=\textwidth]{dependency/2D_surr-low-offwind-onwind.pdf}
    \end{subfigure}
    \begin{subfigure}[t]{0.45\textwidth}
        \centering
        \caption{hydrogen and battery storage}
        \label{fig:dependencies:hb}
        \includegraphics[width=\textwidth]{dependency/2D_surr-low-H2-battery.pdf}
    \end{subfigure}
    }
    \noindent\makebox[\textwidth]{
    \begin{subfigure}[t]{0.65\textwidth}
        \vspace{1cm}
     \centering
     \caption{minimal onshore wind with 8\% system cost slack}
     \label{fig:nearviolin:onwind}
     \includegraphics[width=\textwidth, trim=0cm .3cm 3.3cm .63cm, clip]{violins/violin-capacities-high-prediction-min-onwind-0.08.pdf}
    \end{subfigure}
    \begin{subfigure}[t]{0.65\textwidth}
        \vspace{1cm}
     \centering
     \caption{minimal transmission expansion with 8\% system cost slack}
     \label{fig:nearviolin:transmission}
     \includegraphics[width=\textwidth, trim=0cm .3cm 3.3cm  .63cm, clip]{violins/violin-capacities-high-prediction-min-transmission-0.08.pdf}
    \end{subfigure}
    }
    \vspace{1cm}
    \caption{
      Space of near-optimal solutions by selected pairs of technologies
      under cost uncertainty (a)-(c). Just like in \cref{fig:fuzzycone}, the
      contour lines depict the overlap of the space of near-optimal alternatives
      across the parameter space. It can be thought of as the cross-section of
      the probabilistic near-optimal feasible space for a given $\epsilon$ in
      two technology dimensions and highlights that the extremes of two
      technologies from \cref{fig:fuzzycone} cannot be achieved simultaneously.
      Plots (d)-(e) show the distribution of total system cost, generation,
      storage, and transmission capacities for two near-optimal search
      directions with $\epsilon=8\%$ system cost slack. The dashed line in the
      transmission line chart indicates today's existing transmission
      capacities.
    }
    \label{fig:dependencies}
\end{figure}

The fuzzy cones from \cref{fig:fuzzycone} look at trade-offs between system cost
and single techologies, assuming that the siting and deployment of other
technologies can be heavily optimised. But as there are dependencies between the
technologies, in \crefrange{fig:dependencies:ws}{fig:dependencies:hb} we
furthermore evaluate trade-offs between technologies for three selected pairs of technologies at
an example fixed system cost increase of $\epsilon=6\%$ for illustration, addressing which
\textit{combinations} of wind and solar capacity, offshore and onshore turbines,
and hydrogen and battery storage are likely to be cost-efficient.

First, \cref{fig:dependencies:ws} addresses constraints between wind and solar.
The upper right boundary exists because building much of both wind and solar
would be too expensive for the given budget. The absence of solutions in the
bottom left corner means that building too little of any wind or solar does not
suffice to generate enough electricity. From the shape and contours, we see a
high chance that building 1000 GW of wind \textit{and} 400 GW of solar is within
6\% of the cost-optimum for the scenarios at hand. On the other hand, building
less than 200 GW of solar and 600 GW of wind is unlikely to yield a low-cost
solution in our model setup. In general, minimising the capacity of both primal
energy sources will shift capacity installations to high-yield locations even if
additional network expansion is necessary and boost the preference for highly
efficient storage technologies. Overall, we can conclude from this that, even
considering combinations of wind and solar, a wide space of low-cost options
exists with moderate to high likelihood, although the range of alternatives is
shown to be more constrained.

The trade-off between onshore wind and offshore wind is illustrated in
\cref{fig:dependencies:oo}. Here, the most certain area is characterised by
building more than 600 GW onshore wind, and less than 250 GW offshore wind
capacity for our electricity-only scenarios. However, there are some solutions
with high substitutability between onshore and offshore wind, shown in the upper
left bulge of the contour plot. Compared to wind and solar, the range of
near-optimal solutions is even more constrained. The key role of energy storage
in a fully renewable system is underlined in \cref{fig:dependencies:hb}. Around
50 GW of power capacity of each is at least needed in any of the considered
cases, while highest likelihoods are attained when building 150 GW of each.

\usubsection{Capacity Distributions at Minimal Onshore Wind and Transmission Grid}{sec:sub:dist2}

The aforementioned contour plots \crefrange{fig:fuzzycone}{fig:dependencies} outline
what is likely possible within specified cost ranges and subject to technology cost uncertainty,
but do not expose
the changes the overall system layout experiences when reaching for the extremes in one technology.
Therefore, we show in \crefrange{fig:nearviolin:onwind}{fig:nearviolin:transmission} how the system-wide capacity distributions vary
compared to the least-cost solutions (\cref{fig:violin}) for two illustrative alternative objectives.
For that, we choose the scenarios with least onshore wind capacity and least transmission expansion
because they are often linked to the social acceptance of energy infrastructures.

\cref{fig:nearviolin:onwind} illustrates that reducing onshore wind capacity is
predominantly compensated for by increased offshore wind generation but also added solar capacities.
The increased focus on offshore wind also leads to a tendency towards more hydrogen storage,
while transmission expansion levels are similarly distributed as for the least-cost solutions.
From \cref{fig:nearviolin:transmission} we can further extract that avoiding transmission expansion entails
more hydrogen storage that compensates balancing in space with balancing in time,
and more generation capacity overall, where resources are distributed to locations with
high demand but weaker capacity factors and more heavily curtailed.



% ------------- unused ---------------
% - These cheap areas should then be targeted with policy (e.g. drive technology learning by subsidies)
% - embed technological learning in optimisation \cite{heuberger_power_2017} \cite{lopion_cost_2019}, remaining uncertainty of learning rate, but "learning by doing" included