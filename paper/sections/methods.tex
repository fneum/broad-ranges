\subsection*{Resource Availability}

\subsubsection*{Lead Contact}

Requests for further information, resources and materials should be directed to
the lead contact, Fabian Neumann
(\href{mailto:f.neumann@tu-berlin.de}{f.neumann@tu-berlin.de}).

\subsubsection*{Materials availability}

Not applicable.

\subsubsection*{Data and Code Availability}

The code to reproduce the experiments as well as results dat including selected
networks and all graphics is available at
\href{https://github.com/fneum/broad-ranges}{github.com/fneum/broad-ranges} and
archived at
\href{https://doi.org/10.5281/zenodo.6641551}{doi:10.5281/zenodo.6641551}. We
also refer to the documentation of PyPSA
(\href{https://pypsa.readthedocs.io}{pypsa.readthedocs.io}) and PyPSA-Eur
(\href{https://pypsa-eur.readthedocs.io}{pypsa-eur.readthedocs.io}).

\subsection*{Method Details}

For the description of our experimental procedures, we first outline how we
obtain least-cost and near-optimal solutions for a given cost parameter set. We
then describe the model of the European power system and define the cost
uncertainties. Finally, we explain how we make use of multi-fidelity surrogate
modelling techniques based on polynomial chaos expansions and find an
experimental design that efficiently covers the parameter space.

\usubsection{Least-Cost Investment Planning}{sec:leastcost}

The objective of long-term power system planning is to minimise the total
annual system costs, comprising annualised capital costs $c_\star$ for investments at locations $i$
in generator capacity $G_{i,r}$ of technology $r$, storage capacity $H_{i,s}$ of technology $s$, and transmission line capacities
$F_{\ell}$, as well as the variable operating costs $o_\star$ for generator dispatch $g_{i,r,t}$:
\begin{align}
    \min_{G,H,F,g} \quad \left\{
        \sum_{i,r}   c_{i,r}  \cdot G_{i,r}  +
        \sum_{i,s}   c_{i,s}  \cdot H_{i,s}  +
        \sum_{\ell}  c_{\ell} \cdot F_{\ell} +
        \sum_{i,r,t} w_t \cdot o_{i,r} \cdot g_{i,r,t}
    \right\}
    \label{eq:objective}
\end{align}
where the snapshots $t$ are weighted by $w_t$ such that their total duration
adds up to one year. The objective is subject to a set of linear constraints that define limits on
(i) the capacities of infrastructure from geographical and technical potentials,
(ii) the availability of variable renewable energy sources for each location and point in time, and
(iii) linearised multi-period optimal power flow (LOPF) constraints including storage consistency equations,
which we describe in more detail in the following.

The capacities of generation, storage and transmission infrastructure are
limited to their geographical potentials from above and existing infrastructure from below:
\begin{align}
    \label{eq:firstA}
    \underline{G}_{i,r}  \leq G_{i,r}  \leq \overline{G}_{i,r}  &\qquad\forall i, r \\
    \underline{H}_{i,s}  \leq H_{i,s}  \leq \overline{H}_{i,s}  &\qquad\forall i, s \\
    \underline{F}_{\ell} \leq F_{\ell} \leq \overline{F}_{\ell} &\qquad\forall \ell
\end{align}

The dispatch of a renewable generator is constrained by
its rated capacity and the time- and location-dependent availability $\overline{g}_{i,r,t}$,
given in per-unit of the generator's capacity:
\begin{align}
    0 \leq g_{i,r,t} \leq \overline{g}_{i,r,t} G_{i,r} \qquad\forall i, r, t
\end{align}
The dispatch of storage units is described by a charge variable $h_{i,s,t}^+$
and a discharge variable $h_{i,s,t}^-$, each limited by the power rating $H_{i,s}$.
\begin{align}
    0 \leq h_{i,s,t}^+ \leq H_{i,s} &\qquad\forall i, s, t \\
    0 \leq h_{i,s,t}^- \leq H_{i,s} &\qquad\forall i, s, t
\end{align}
The energy levels $e_{i,s,t}$ of all storage units are linked to the dispatch by
\begin{align}
    e_{i,s,t} =\: & \eta_{i,s,0}^{w_t} \cdot e_{i,s,t-1} + w_t \cdot h_{i,s,t}^\text{inflow} - w_t \cdot h_{i,s,t}^\text{spillage} & \quad\forall i, s, t \nonumber \\
    & + \eta_{i,s,+} \cdot w_t \cdot h_{i,s,t}^+ - \eta_{i,s,-}^{-1} \cdot w_t \cdot h_{i,s,t}^-.
\end{align}
Storage units can have a standing loss $\eta_{i,s,0}$, a charging efficiency $\eta_{i,s,+}$, a discharging efficiency $\eta_{i,s,-}$,
natural inflow $h_{i,s,t}^\text{inflow}$ and spillage $h_{i,s,t}^\text{spillage}$.
The storage energy levels are assumed to be cyclic and are constrained by their energy capacity
\begin{align}
    e_{i,s,0} = e_{i,s,T} &\qquad\forall i, s \\
    0 \leq e_{i,s,t} \leq \overline{T}_s \cdot H_{i,s} &\qquad\forall i, s, t.
\end{align}
To reduce the number of decisison variables, we link the energy capacity to
power ratings with a technology-specific parameter $\overline{T}_s$ that
describes the maximum duration a storage unit can discharge at full power
rating. Owing to the absence of large thermal power plants in our fully
renewable scenarios, unit commitment constraints regarding the ramping, start-up
and shut-down of generators are not considered.

Kirchhoff's Current Law (KCL) requires local generators and storage units as well as
incoming or outgoing flows $f_{\ell,t}$ of incident transmission lines $\ell$
to balance the inelastic electricity demand $d_{i,t}$ at each location $i$ and snapshot $t$
\begin{align}
    \sum_r g_{i,r,t} + \sum_s h_{i,s,t} + \sum_\ell K_{i\ell} f_{\ell,t} = d_{i,t} \qquad\forall i,t,
\end{align}
where $K_{i\ell}$ is the incidence matrix of the network.

Kichhoff's Voltage Law (KVL) imposes further constraints on the flow of AC lines.
Using linearised load flow assumptions, the voltage angle difference around every closed cycle in the
network must add up to zero. We formulate this constraint using a cycle basis $C_{\ell c}$
of the network graph where the independent cycles $c$ are expressed as
directed linear combinations of lines $\ell$.\cite{cycleflows}
This leads to the constraints
\begin{align}
    \sum_\ell C_{\ell c} \cdot x_\ell \cdot f_{\ell,t} = 0 \qquad\forall c,t
    \label{eq:kvl}
\end{align}
where $x_\ell$ is the series inductive reactance of line $\ell$.
Controllable HVDC links are not affected by this constraint.

Finally, all line flows $f_{\ell,t}$ must be operated within their nominal capacities $F_\ell$
\begin{align}
    \abs{f_{\ell,t}} \leq \overline{f}_{\ell} F_{\ell} & \qquad\forall \ell, t,
    \label{eq:lastA}
\end{align}
where $\overline{f}_\ell$ acts as a per-unit buffer capacity
to protect against the outage of single circuits.

This problem is implemented in the open-source tool PyPSA\cite{pypsa} and is solved by Gurobi.
Note, that it assumes perfect foresight for a single reference year based on which capacities are optimised.
It does not include pathway optimisation, nor aspects of reserve power, or system stability.
Changes of line expansion to line impedance are ignored.

\usubsection{Finding Near-Optimal Alternatives}{sec:nearoptimal}

Using the least-cost solution as an anchor, we use the
$\epsilon$-constraint method from multi-objective optimisation
to find near-optimal feasible solutions.\cite{nearoptimal,mavrotas_effective_2009}
For notational brevity, let $c^\top x$ denote the linear objective function \cref{eq:objective}
and $Ax\leq b$ the set of linear constraints \crefrange{eq:firstA}{eq:lastA}
in a space of continuous variables,
such that the minimised system cost can be represented by
\begin{equation}
    C = \min_x\left\{c^\top x \mid Ax\leq b\right\}.
\end{equation}

% one technology

We then encode the original objective as a constraint
such that the cost increase is limited to a given $\epsilon$.
In other words, the feasible space is cut to solutions that
are at most $\epsilon$ more expensive than the least-cost solution.
Given this slack, we can formulate alternative search directions in the objective.
For instance, we can seek to minimise or maximise the sum of solar installations $x_s \subseteq x$ with
\begin{align}
    \label{eq:xs_max}
    \underline{x_s} = \min_{x_s}\left\{\: 1^\top x_s \mid Ax\leq b,\quad c^\top x\leq (1+\epsilon)\cdot C \:\right\} \\
    \label{eq:xs_min}
    \overline{x_s} = \max_{x_s}\left\{\: 1^\top x_s \mid Ax\leq b,\quad c^\top x\leq (1+\epsilon)\cdot C \:\right\}.
\end{align}
To draw a full picture of the boundaries of the near-optimal feasible space,
we systematically explore the extremes of various technologies:
we both minimise and maximise the system-wide investments in
solar, onshore wind, offshore wind, any wind, hydrogen storage, and battery storage
capacities, as well as the total volume of transmission network expansion.
Evaluating each of these technology groups for
different cost deviations $\epsilon \in \{1\%,2\%,4\%,6\%,8\%\}$
allows us to observe how the degree of freedom regarding investment decisions
rises as the optimality tolerance is increased, both at lower and upper ends.
The boundaries delineate Pareto frontiers on which no criterion,
neither reducing system cost nor extremising the capacity of a technology,
can be improved without depressing the other.
By arguments of convexity, these extremes even define limits
within which all near-optimal solutions are contained.
Moreover, although this scheme primarily studies aggregated capacities,
the solutions are spatially explicit, and we can inspect for each case
how the capacities of each technology are distributed within the network.


% two technologies

The near-optimal analysis above only explores the extremes of one technology at
a time, i.e.~one direction in the feasible space. But actually the space of
attainable solutions within $\epsilon$ of the cost-optimum is multi-dimensional.
To further investigate trade-offs between multiple technologies, in addition to
the $\epsilon$-constraint and the objective to extremise capacities of a
particular technology, we formulate a constraint that fixes the capacity of
another technology within its bounds for a given $\epsilon$. For instance, we
search for the minimum amount of wind capacity $x_w \subseteq x$ given that a
certain amount of solar is built
\begin{equation}
    \label{eq:2D}
    \underline{x_w} = \min_{x_w}\left\{\:1^\top x_w \mid Ax\leq b,\quad c^\top x\leq (1+\epsilon)\cdot C, \quad 1^\top x_s = \underline{x_s} + \alpha \cdot (\overline{x_s}-\underline{x_s}) \:\right\},
\end{equation}
and correspondingly for the maximum $\overline{x_w}$. The $\alpha$ denotes the
relative position within the near-optimal range of the second technology at
given $\epsilon$, in this case the solar capacities. For example, at
$\alpha=0\%$ we look for the least wind capacity given that minimal solar
capacities are built for the given $\epsilon$. An alternative but more complex
approach to spanning the space of near-optimal solutions in multiple dimensions
at a time using a quick hull algorithm was presented by Pedersen et
al.~\cite{pedersen_modeling_2020}

% ----------------- unused --------------------------
% extremes of different technologies cannot be realised simultaneously.
% This is because technologies that are not part of the objective
% will act to support the active search direction.
% For instance, lowering wind capacities will be compensated for
% by higher solar capacities to satisfy the total energy demand.
% Thus, extremising one technology narrows the flexibility of a substitute technology.

Due to computational constraints, we focus on technologies which are assumed to
lend themselves to substitution and limit the corresponding analysis to a single
illustrative cost increase level of $\epsilon=6\%$. The same methodology can be
applied to any other value for $\epsilon$. We consider the three pairs, (i) wind
and solar, (ii) offshore and onshore wind, (iii) hydrogen and battery storage,
by minimising and maximising the former while fixing the latter at positions
$\alpha \in \{0\%,25\%,50\%,75\%,100\%\}$ within the respective near-optimal
range.

\usubsection{Open Electricity Sector Model PyPSA-Eur}{sec:inputs}

\begin{figure}
    \centering
        \begin{tabular}{cc}
            \footnotesize (a) low-fidelity: & \footnotesize (b) high-fidelity: \\
            \footnotesize 37 nodes and 4-hourly & \footnotesize 128 nodes and 2-hourly \\
            \includegraphics[width=0.49\textwidth]{map37.pdf} &
            \includegraphics[width=0.49\textwidth]{map128.pdf} \\
            \includegraphics[width=0.49\textwidth]{timeseries37.pdf} &
            \includegraphics[width=0.49\textwidth]{timeseries128.pdf} \\
        \end{tabular}
    \caption[Spatial and temporal resolution of the low and high fidelity model]{Spatial and temporal resolution of the low and high fidelity model.
    Green lines represent controllable HVDC lines. Red lines represent HVAC lines.
    Examples for capacity factors for wind and solar are shown for four days in March
    at the northernmost node in Germany, alongside the normalised load profile.}
    \label{fig:pypsaeur}
\end{figure}

% general

The instances of the coordinated capacity expansion problem
 (see \nameref{sec:leastcost}) are based on \mbox{PyPSA-Eur}, which is an open model of
 the European power transmission system that combines high spatial and temporal
 resolution.\cite{pypsaeur} Because it only uses open data and every processing
 step is defined in a workflow,\cite{snakemake} we achieve a high level of
 transparency and reproducibility. In the following, we outline the main
 features and configurations, and refer to the supplementary material and Hörsch
 et al.~\cite{pypsaeur} for more details.

\paragraph{Scenario}
We target a fully renewable electricity system based on variable resources such
as solar photovoltaics, onshore wind and offshore wind, that has not carbon
emissions. We pursue a greenfield approach subject to a few notable exceptions.
The existing hydro-electric infrastructure (run-of-river, hydro dams,
pumped-storage) is included but not considered to be extendable due to assumed
geographical constraints. Furthermore, the existing transmission infrastructure
can only be reinforced continuously but may not be removed. In addition to
balancing renewables in space with transmission networks, the model includes
storage options at each node to balance renewables in time. We consider two
extendable storage technologies: battery storage representing short-term storage
suited to balancing daily fluctuations and hydrogen storage which exemplifies
long-term synoptic and seasonal storage. We do not consider any further
sector-coupling technologies or demand-side flexibilities, but expect that these
could substitute for some of the storage requirements.

\paragraph{Spatial and Temporal Resolution}
Since the spatial and temporal resolution strongly affects the size of the
optimisation problem, running the model at full resolution is computationally
infeasible. In our analysis, we therefore make use of two levels of aggregation,
reflecting a compromise between the computational burden incurred by
high-resolution models and the growing inaccuracies regarding transmission
bottlenecks and resource distribution in low-resolution models. We consider a
low-fidelity model with 37 nodes at a 4-hourly resolution for a full year that
models power flow via a transport model (i.e.~excluding KVL of \cref{eq:kvl})
and a high-fidelity model with 128 nodes at a 2-hourly resolution that is
subject to linearised load flow constraints (\cref{fig:pypsaeur}). More
information on how the results of two models with varying resolution are merged
is provided in the section \nameref{sec:multifidelity}.

% existing infrastructure (hydro and transmission)

\paragraph{Transmission Grid and Hydro-Electricity}
The topology of the European transmission network is retrieved from the ENTSO-E
transparency map and includes all lines at and above 220 kV. Capacities and
electrical characteristics of transmission lines and substations are inferred
from standard types for each voltage level, before they are transformed to a
uniform voltage level. For each line, $N-1$ security is approximated by limiting
the line loading to 70\% of its nominal rating. The dataset further includes
existing high-voltage direct current (HVDC) links and planned projects from the
Ten Year Network Development Plan (TYNDP). Existing run-of-river, hydro-electric
dams, pumped-hydro storage plants are retrieved from
\textit{powerplantmatching}, a merged dataset of conventional power plants.

% renewable generation potentials

\paragraph{Renewable Energy Potentials}
Eligible areas for developing renewable infrastructure are calculated
per technology and the grid nodes' Voronoi cells, assuming wind and solar installations always connect to the closest substation.
How much wind and solar capacity may be built at a location is constrained by
eligible codes of the CORINE land use database and is further restricted by distance criteria,
allowed deployment density, and the natural protection areas specified in the NATURA 2000 dataset.
Moreover, offshore wind farms may not be developed at sea depths exceeding 50 metres,
as indicated by the GEBCO bathymetry dataset.

% renewable generation and demand time series

\paragraph{Renewables and Demand Time Series}
The location-dependent renewables availability time series are generated based
on two historical weather datasets for the year 2013, which is an average year
in terms of wind and solar availability. We retrieve wind speeds, run-off and
surface roughness from the ERA5 reanalysis dataset and use the satellite-aided
SARAH-2 dataset for the direct and diffuse surface solar irradiance. Models for
wind turbines, solar panels, and the inflow into the basins of hydro-electric
dams convert the weather data to hourly capacity factors and aggregate these to
each grid node. Historical country-level load time series are taken from ENTSO-E
statistics and are heuristically distributed to each grid node to 40\% by
population density and to 60\% by gross domestic product.

\usubsection{Technology Cost Uncertainty}{sec:uncertainty}

\begin{SCtable}
    \begin{small}
        \begin{tabular}{cccc}
            \toprule
            Technology & Lower CAPEX & Upper CAPEX & Unit  \\ \midrule
            Onshore Wind & 800 & 1190 & EUR/kW \\
            Offshore Wind & 1420 & 1950 & EUR/kW \\ % no connection cost
            Solar & 420 & 620 & EUR/kW \\
            Battery & 316 & 1306 & EUR/kW \\
            Hydrogen & 668 & 2002 & EUR/kW \\ \bottomrule
            % FOM and lifetime may also vary!
        \end{tabular}
    \end{small}
    \caption[Technology Cost Uncertainty]{Technology cost uncertainty using optimistic and pessimistic assumptions from the Danish Energy Agency.\cite{DEA}}
    \label{tab:costuncertainty}
\end{SCtable}

% General

Uncertainty of technology cost projections is driven by two main factors:
unknown learning rates (i.e.~how quickly costs fall as more capacity is built) and
unclear deployment rates (i.e.~how much capacity will be built in the future).\cite{gritsevskyi_modeling_2000,yeh_review_2012}
As modelling technological learning endogeneously is computationally challenging due to the nonconvexity it entails,\cite{heuberger_power_2017,mattsson_learning_2019,lisa_learning2022}
% While uncertainty about the latter could generally be addressed by modelling endogeneous
% learning in an approach that includes multiple investment periods,
% the exponential influence of the learning rate raises the computational complexity immensely
technology cost uncertainty is typically defined exogenously by an interval within which costs may vary
and a distribution that specifies which segments are more probable.

% Ranges

Ranges of cost projections are best chosen as wide as possible to avoid
excluding any plausible scenarios.
\cite{moret_characterization_2017,mccollum_energy_2020} When uncertainty has
been considered in the literature, cost assumptions have commonly been modelled
to vary between $\pm20\%$ and $\pm65\%$ depending on the technology's maturity.
\cite{moret_characterization_2017,shirizadeh_how_2019,pizarro-alonso_uncertainties_2019,li_using_2020,trondle_trade-offs_2020}
In this study, we consider uncertainty regarding the annuities of onshore wind,
offshore wind, solar PV, battery and hydrogen storage systems. The latter
comprises the cost of electrolysis, cavern storage, and fuel cells. For solar PV
we assume an even split between utility-scale PV and residential rooftop PV.
Evaluating uncertainties based on annuities has a distinct advantage. They can
be seen to simultaneously incorporate uncertainties about the overnight
investments, fixed operation and maintenance costs, their lifetime, and the
discount rate, since multiple combinations lead to the same annuity. We built
the uncertainty ranges presented in \cref{tab:costuncertainty} from the
optimistic and pessimistic technology cost and lifetime projections for the year
2050 from the Danish Energy Agency, which correspond to 90\% confidence
intervals.\cite{DEA} In cases where no uncertainty ranges were provided
for the year 2050, such as for rooftop PV, projections for the year 2030
define the upper end of the uncertainty interval.

% Distributions

Distributions of cost projections have been assumed to follow
normal\cite{mavromatidis_uncertainty_2018} or triangular\cite{li_using_2020}
distributions. But independent uniform distributions are the most prevalent
assumption.\cite{moret_characterization_2017,moret_robust_2016,shirizadeh_how_2019,trondle_trade-offs_2020,pilpola_analyzing_2020,Li2017,Trutnevyte2013,lopion_cost_2019}
This approach is backed by the maximum entropy approach,
\cite{trondle_trade-offs_2020} which states that given the persistent lack of
knowledge about the distribution the independent uniform distribution, that
makes fewest assumptions, is most appropriate. Although the assumed independence
may neglect synergies between technologies, for example, between offshore and
onshore wind turbine development, we follow the literature by assuming that the
cost are independent and uniformly distributed within the ranges specified in
\cref{tab:costuncertainty}.

\usubsection{Surrogate Modelling with Polynomial Chaos Expansion}{sec:surrogate}

% problem

Searching for least-cost solutions (see \nameref{sec:leastcost}) and many
associated near-optimal alternatives (see \nameref{sec:nearoptimal}) of a highly
resolved power system model (see \nameref{sec:inputs}) on its own is already
labour-intensive from a computational perspective. Repeating this search for a
large variety of cost assumptions (see \nameref{sec:uncertainty}), to be able to
make statements about the robustness of investment flexibility near the optimum
under uncertainty, adds another layer to the computational burden.

% idea of surrogate models

Surrogate models offer a solution for such cases, where the outcome of the
original model cannot be obtained easily. Surrogate names are also known by
names such as approximation models, response surface methods, metamodels and
emulators. In contrast to the full model, they only imitate the input/output
behaviour for a selection of aggregated outputs, but take much less time to
compute.\cite{palar_multi-fidelity_2016} Like other machine learning
techniques, they generalise from a training dataset that comprises only a
limited number of samples. As surrogate models interpolate gaps in the parameter
space that are not contained in the sample set, which would otherwise be
computationally expensive to fill, they are well suited to use cases such as
parameter space exploration and sensitivity analysis.

% this paper

Consequently, in this paper we make use of surrogate models that map the cost of
onshore wind, offshore wind, solar, hydrogen, and battery storage
(\cref{tab:costuncertainty}) onto a selection of eight system-level outputs.
These are the total system cost and the installed onshore wind, offshore wind,
solar, hydrogen, battery, and transmission network capacities. We construct
surrogate models for least-cost and near-optimal solutions separately for each
system cost slack, search direction, fixed total capacity, and output variable.
This results in a collection of 808 individual surrogate models based on 101
solved optimisation problems per set of cost assumptions. The method we choose
from an abundance of alternatives is based on polynomial chaos expansion (PCE).
\cite{sudret_global_2008,fajraoui_optimal_2017,gratiet_metamodel-based_2015} We
select this approach because the resulting approximations allow efficient
analytical statistical evaluation\cite{sudret_global_2008} and can conveniently
combine training data from variously detailed models.\cite{palar_multi-fidelity_2016}

% general idea

The general idea of surrogate models based on PCE is to
represent uncertain model outputs as a linear combination of orthogonal basis functions
of the random input variables weighted by deterministic coefficients.\cite{muhlpfordt_uncertainty_2020}
It is a Hilbert space technique that works in principle analogously to decomposing a periodic signal into its Fourier components.\cite{muhlpfordt_uncertainty_2020}
Building the surrogate model consists of the following steps:
(i) sampling a set of cost projections from the parameter space,
(ii) solving the least-cost or near-optimal investment planning problem for each sample,
(iii) selecting an expansion of orthogonal polynomials within the parameter space,
(iv) performing a regression to calculate the polynomial coefficients, and ultimately
(v) using the model approximation for statistical analysis.
In the following, we formalise this approach mathematically,
which we implemented using the \textit{chaospy} toolbox,\cite{feinberg_chaospy_2015}
and elaborate on individual aspects in more detail.

We start by defining the vector of random input variables as
\begin{equation}
    \x = \{\mathsf{x}_1, \dots , \mathsf{x}_m\}
\end{equation}
that represents the $m$ uncertain cost projections.
Further, we let
\begin{equation}
    \mathsf{y} = f(\x)
\end{equation}
describe how the uncertainty of inputs $\x$ propagates
through the computationally intensive model $f$
(i.e.~the solving of a large capacity expansion problem)
to the outputs $\mathsf{y} \in \mathbb{R}$.
% For simplicity, we limit the subsequent explanation to a scalar output,
% but it is straightforward to expand the theory to vectorised outputs.

We can represent the computational model $f$ with its polynomial chaos expansion
\begin{equation}
    \mathsf{y} = f(\x) = \sum_{\ba \,\in\, \mathbb{N}^m} r_{\ba} \psi_{\ba}(\x),
    \label{eq:pce}
\end{equation}
where $\psi_\ba$ denotes multivariate orthogonal polynomials that form
a Hilbertian basis and $r_\ba \in \mathbb{R}$ are the corresponding polynomial coefficients.\cite{sudret_global_2008} The multiindex $\ba = \{\alpha_1,\dots,\alpha_m\}$
denotes the degree of the polynomial $\psi_\ba$ in each of the $m$ random input variables $\mathsf{x}_i$.
As \cref{eq:pce} features an infinite number of unknown coefficients,
it is common practice to approximate by truncating the expansion to get a finite number of coefficients
\begin{equation}
    f(\x) \approx f'(\x) = \sum_{\ba \,\in\, \cA^{m,p}} r_\ba \psi_\ba(\x).
\end{equation}
In the standard truncation
scheme,\cite{gratiet_metamodel-based_2015,sudret_global_2008} all polynomials in
$m$ input variables (i.e.~cost uncertainties) where the total degree is less
than a user-defined $p$ are selected. We can write this as a set of indices
\begin{equation}
    \label{eq:ca}
    \cA^{m,p} = \left\{\ba \in \mathbb{N}^m \,:\, \abs{\ba} \leq p\right\},
\end{equation}
where $\abs{\ba} = \sum_{i=1}^m \alpha_i$.
Given the joint distribution of cost uncertainties of $\x$ and a maximum degree,
a suitable collection of orthogonal polynomials can be constructed
using a three terms recurrence algorithm.\cite{feinberg_chaospy_2015}
The cardinality of the truncated PCE,
\begin{equation}
    q = \card \cA^{m,p} = \left(\begin{matrix}
        m+p \\
        p
    \end{matrix}\right) = \frac{(m+p)!}{m!p!},
    \label{eq:cardinality}
\end{equation}
indicates the number of unknown polynomial coefficients.

We determine these coefficients by a regression based on
a set of cost parameter samples
and the corresponding outputs,
\begin{equation}
    \mathcal{X} = \set{ \bm x^{(1)},\dots,\bm x^{(n)} } \quad\text{and}\quad
    \mathcal{Y} = \set{ f\left(\bm x^{(1)}\right),\dots,f\left(\bm x^{(n)}\right) }.
\end{equation}
Using this training dataset, we minimise the least-square residual of the polynomial approximation across all observations.
We add an extra $L_1$ regularisation term, % LASSO regression
that induces a preference for fewer non-zero coefficients, and solve
\begin{equation}
    \hat{\bm{r}} = \argmin_{\bm{r} \,\in\, \mathbb{R}^q} \left[ \frac{1}{n} \sum_{i=1}^n \left(
        f\left(\bm x^{(i)}\right) - \sum_{\ba \,\in\, \cA^{m,p}} r_\ba \psi_\ba\left(\bm x^{(i)}\right)
        \right)^2  + \lambda \norm{\bm{r}}_1 \right],
        \label{eq:regression}
    \end{equation}
where we set the regularisation penalty to $\lambda=0.005$.
This results in a sparse PCE that has proven to
improve approximations in high-dimensional uncertainty spaces
and to reduce the required number of samples for comparable approximation errors.\cite{gratiet_metamodel-based_2015}
Knowing the optimised regression coefficients, we can now assemble the complete surrogate model
\begin{equation}
    \mathsf{y} = f(\x) \approx f'(\x) = \sum_{\ba \,\in\, \cA^{m,p}} \hat{r}_\ba \psi_\ba (\x).
\end{equation}

\usubsection{Multifidelity Approach in Surrogate Modelling}{sec:multifidelity}

% idea

To construct a sufficiently precise PCE-based surrogate model, it is desirable
to base it on many samples from a high-fidelity model. However, this is likely
prohibitively time-consuming. On the other hand, relying only on samples from a
low-fidelity model may be too inaccurate.\cite{ng_multifidelity_2012} For
example, an investment model that features only a single node per country will
underestimate transmission bottlenecks and regionally uneven resource or demand
distribution. In the section on \nameref{sec:inputs} we already alluded to using two models with
varying spatial and temporal resolution in this paper. We integrate both in a
multi-fidelity approach,\cite{ng_multifidelity_2012,palar_multi-fidelity_2016}
and demonstrate how we can simultaneously avail of high coverage of the
parameter space by sampling the simpler model many times, and the high
spatio-temporal detail yielded by fewer more complex model runs.

% method

The idea of the multi-fidelity approach is to build a corrective surrogate model $f_\Delta'(\x)$ for the
error of the low-fidelity model $f_\ell$ compared to the high-fidelity model $f_h$
\begin{equation}
    f_\Delta(\x) = f_h(\x) - f_\ell(\x),
\end{equation}
and add it to a surrogate model of the low-fidelity model to approximate the behaviour of the
high-fidelity model
\begin{equation}
    f_h'(\x) = f_\ell'(\x) + f_\Delta'(\x).
\end{equation}
Typically, the corrective PCE rectifies only the lower order effects of the
low-fidelity surrogate model (e.g.~linear effects of an individual technology's
cost).\cite{palar_multi-fidelity_2016} The advantage is that this way the
correction function can be determined based on fewer samples analogous to the
previous section \nameref{sec:surrogate}. To sample the errors, it is only
required that the high-fidelity samples are a subset of the low-fidelity
samples, e.g.
\begin{equation}
    \mathcal{X}_h = \set{ \bm{x}^{(1)}, \dots, \bm{x}^{(n_h)}} \quad\text{and}\quad
    \mathcal{X}_\ell = \set{ \bm{x}^{(1)}, \dots, \bm{x}^{(n_h)}, \dots, \bm{x}^{(n_\ell)}},
\end{equation}
which we can easily guarantee by using deterministic low-discrepancy series
in the experimental design (see \nameref{sec:sampling}).
With $p_\Delta < p_\ell$ and consequently $\cA_\Delta \subset \cA_\ell$,
the multi-fidelity surrogate model can be written as a combination of low-fidelity
and corrective polynomial coefficients
\begin{equation}
    f_h' (\x) = \sum_{\ba\,\in\,\cA_\ell^{m,p_\ell} \,\cap\, \cA_\Delta^{m,p_\Delta}}
    \left(
     r_{\ell,\ba} + r_{\Delta,\ba}
    \right) \psi_\ba(\x) +
    \sum_{\ba\,\in\,\cA_\ell^{m,p_\ell} \,\setminus\, \cA_\Delta^{m,p_\Delta}}
    r_{\ell,\ba} \psi_\ba(\x).
\end{equation}

% this paper

In this work, we apply a multi-fidelity surrogate model that considers
effects up to order three observed in the low-fidelity model. These are then corrected with linear terms derived from insights from the high-fidelity model.
We justify this choice by experimentation in the \nameref{sec:validation},
by testing against other typical choices between orders one to five.\cite{gratiet_metamodel-based_2015}
Given the polynomial expansion order, the remaining question is how many samples are necessary to attain an acceptable approximation.

\usubsection{Experimental Design for Surrogate Modelling}{sec:sampling}

% competition/benchmark: \cite{trondle_trade-offs_2020}
% - high fidelity: 10 samples, 400 nodes, 4-hourly;
% - low fidelity: 150 samples, 25 nodes, 4-hourly; no DC power flow
% uncertain parameters
%12 in \cite{trondle_trade-offs_2020}
%36 in \cite{pilpola_analyzing_2020}
%5 in \cite{shirizadeh_how_2019}

% general

The experimental design covers strategies to find sufficiently high coverage
of the parameter space at low computational cost.\cite{fajraoui_optimal_2017,usher_value_2015}
It deals with how many samples are drawn and what sampling method is used.

% sampling method

Traditional Monte-Carlo sampling with pseudo-random numbers is known to
possess slow convergence properties, % 1 / \sqrt{n}
especially in high-dimensional parameter spaces.
So-called low-discrepancy series can greatly improve on random sampling.
Because they are designed to avoid forming large gaps and clusters,
these deterministic sequences efficiently sample from the parameter space.\cite{fajraoui_optimal_2017}
Thus, we choose to draw our samples from a low-discrepancy Halton sequence.

% how many samples

For the question about how many samples should be drawn,
we resort to the oversampling ratio (OSR) as a guideline.
The OSR is defined as the ratio between the number of samples
and the number of unknown coefficients.\cite{palar_multi-fidelity_2016}
The literature recommends values between two and three.\cite{hosder2007,palar_multi-fidelity_2016,fajraoui_optimal_2017,gratiet_metamodel-based_2015}
In other words, for a sufficiently accurate approximation,
there should be significantly more samples than unknown coefficients.
If the OSR is lower, the regression is prone to the risk of overfitting.
On the other hand, a high OSR may lead to a very coarse approximation.\cite{palar_multi-fidelity_2016}

According to \cref{eq:cardinality}, targeting an OSR of two
and considering the five uncertain technology cost parameters (\cref{tab:costuncertainty}),
approximating linear effects would require at least 12 samples, whereas cubic relations
would already need 112 samples. Even 504 samples would be necessary to model the dynamics of order 5.
To investigate the quality of different PCE orders
and retain a validation dataset,
we draw 500 samples for the low-fidelity model.
Due to the computational burden carried by the high-fidelity models,
we settle on a linear correction in advance, such that
15 samples for the high-fidelity model are acceptable.
In combination with 101 least-cost and near-optimal optimisation
runs for each sample, this setup results in a total number of
50,500 runs of the low-fidelity model and
1,515 runs of the high-fidelity model.
On average a single high-fidelity model run took 20 GB of memory and 5 hours to solve.
Each low-fidelity model run on average consumed 3 GB of memory and completed within 5 minutes.
This setup profits tremendously from parallelisation as it involves
numerous independent optimisation runs.
Moreover, it would have been infeasible to carry out without high-performance computing.

\usubsection{Validation of the Surrogate Models}{sec:validation}

\begin{figure}
    % \noindent\makebox[\textwidth]{
        \begin{subfigure}[t]{.48\textwidth}
            \centering
            \caption{number of samples}
            \label{fig:error:samples}
            \includegraphics[height=.6\textwidth]{error/error-r2-vs-samples-order-3-sklearn.pdf} \\
            \includegraphics[height=.6\textwidth]{error/error-mape-vs-samples-order-3-sklearn.pdf} \\
            \includegraphics[height=.6\textwidth]{error/error-mae-vs-samples-order-3-sklearn.pdf} \\
            \includegraphics[height=.6\textwidth]{error/error-rmse-vs-samples-order-3-sklearn.pdf}
        \end{subfigure}
        \begin{subfigure}[t]{.48\textwidth}
            \centering
            \caption{polynomial order}
            \label{fig:error:poly}
            \includegraphics[height=.6\textwidth]{error/error-r2-vs-order-sklearn.pdf} \\
            \includegraphics[height=.6\textwidth]{error/error-mape-vs-order-sklearn.pdf} \\
            \includegraphics[height=.6\textwidth]{error/error-mae-vs-order-sklearn.pdf} \\
            \includegraphics[height=.6\textwidth]{error/error-rmse-vs-order-sklearn.pdf}
        \end{subfigure}
    % }
    \caption[Cross-validation errors]{Cross-validation errors by output for varying sample sizes and polynomial orders
    of least-cost low-fidelity surrogate models.}
    \label{fig:error}
\end{figure}

We justify the use of surrogate modelling by cross-validation.
Out of the 500 low-fidelity samples,
100 samples are not used in the regression.
This validation dataset is unknown to the surrogate model and is
consulted to assess the approximation's quality.
Because the high-fidelity sample size is limited and
approximating near-optimal solutions is not assumed to fundamentally differ,
we base the validation on low-fidelity least-cost solutions only.
We experimentally evaluate the approximation errors between predicted and observed data
for different combinations of polynomial order and sample size
to decide on a suitable parameterisation.
We present the coefficient of determination (R$^2$)
for the variance captured,
the mean absolute (percentage) errors (MAE/MAPE) for absolute and relative deviations,
and the root mean squared error (RMSE).

Regarding the number of samples required, \cref{fig:error:samples}
foremost illustrates that, given enough samples, we achieve
average relative errors of less than 4\% for most output variables.
This is comparable to the cross-validation errors from Tröndle et al.~\cite{trondle_trade-offs_2020}
at rates below 5\%.
Only for offshore wind and battery storage, we observe larger errors.
However, this can be explained by a distortion of the relative measure
when these technologies are hardly built for some cost projections.
On the contrary, the prediction of total system costs is remarkably accurate.
\cref{fig:error:samples} also demonstrates that for a polynomial order of 3,
we gain no significant improvement with more than 200 samples.
In fact, thanks to the regularisation term used in the regression,
we already attain acceptable levels of accuracy with as few as 50 samples.
Moreover, the high R$^2$ values underline that the surrogate model can explain
most of the output variance.

Regarding the polynomial order, \cref{fig:error:poly} shows that
an order of 2 and below may be too simple to capture the
interaction between different parameters. On the other hand,
an order of 4 and above yields no improvement and,
were it not for the moderating regularisation term,
would even result in a loss of generalisation properties due to overfitting.
As higher-order approximations require significantly more samples,
an order of 3 appears to be a suitable compromise
to limit the computational burden.
