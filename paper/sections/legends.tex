\begin{figure}[H]
    \caption{ {\bf Distribution of system cost, generation, storage, and transmission
      in least-cost solutions.} The dashed line in the transmission line chart
      indicates today's existing transmission capacities for comparison. Violin
      range is limited to the range of observed data. Boxplots show median,
      interquartile range, and upper/lower quartile $\pm 1.5$ times the
      interquartile range. }
    \label{fig:violin}
\end{figure}


\begin{figure}[H]
    \caption{ { \bf Sensitivity of capacities towards their own technology cost.} The
      median (Q50) alongside the 5\%, 25\%, 75\%, and 95\% quantiles (Q5--Q95)
      display the sensitivity subject to the uncertainty induced by other cost
      parameters. Dots represent samples of the high-fidelity model runs. }
    \label{fig:sensitivity}
\end{figure}

\begin{figure}[H]
    \caption{ {\bf First-order and total Sobol indices.} These
      sensitivity indices attribute output variance to random input variables
      and reveal which inputs the outputs are most sensitive to. The first-order
      Sobol indices quantify the share of output variance due to variations in
      one input parameter alone. The total Sobol indices further include
      interactions with other input variables. Total Sobol indices can be
      greater than 100\% if the contributions are not purely additive. }
    \label{fig:sobol}
\end{figure}


\begin{figure}[H]
    \caption{ { \bf Space of near-optimal solutions by technology under cost
    uncertainty. } For each technology and cost sample, the minimum and maximum
    capacities obtained for increasing cost penalties $\epsilon$ form a cone of
    an upper and a lower Pareto front, starting from a common least-cost
    solution. By arguments of convexity, the capacity ranges contained by the
    cone can be near-optimal and feasible, given a degree of freedom in the
    other technologies. From optimisation theory, we know that the cones widen
    up for increased slacks. As we consider technology cost uncertainty, the
    cone will look slightly different for each sample. The contour lines
    represent the frequency a solution is inside the near-optimal cone over the
    whole parameter space. This is calculated from the overlap of many cones,
    each representing a set of cost assumptions. Due to discrete sampling points
    in the $\epsilon$-dimension, the plots further apply quadratic interpolation
    and a Gaussian filter for smoothing. }
    \label{fig:fuzzycone}
\end{figure}


\begin{figure}[H]
    \caption{ {\bf Space of near-optimal solutions by selected pairs of technologies
      under cost uncertainty (a)-(c).} Just like in \cref{fig:fuzzycone}, the
      contour lines depict the overlap of the space of near-optimal alternatives
      across the parameter space. It can be thought of as the cross-section of
      the probabilistic near-optimal feasible space for a given $\epsilon$ in
      two technology dimensions and highlights that the extremes of two
      technologies from \cref{fig:fuzzycone} cannot be achieved simultaneously.
      Plots (d)-(e) show the distribution of total system cost, generation,
      storage, and transmission capacities for two near-optimal search
      directions with $\epsilon=8\%$ system cost slack. The dashed line in the
      transmission line chart indicates today's existing transmission
      capacities. }
    \label{fig:dependencies}
\end{figure}

\begin{figure}[H]
    \caption{ {\bf Spatial and temporal resolution of the low and high fidelity model. }
    Green lines represent controllable HVDC lines. Red lines represent HVAC
    lines. Examples for capacity factors for wind and solar are shown for four
    days in March at the northernmost node in Germany, alongside the normalised
    load profile.}
    \label{fig:pypsaeur}
\end{figure}

\begin{table}[H]
    \begin{small}
        \begin{tabular}{lrrl}
            \toprule
            Technology & Lower CAPEX & Upper CAPEX & Unit  \\ \midrule Onshore
            Wind & 800 & 1190 & EUR/kW \\
            Offshore Wind & 1420 & 1950 & EUR/kW \\
            Solar & 420 & 620 & EUR/kW \\
            Battery & 316 & 1306 & EUR/kW \\
            Hydrogen & 668 & 2002 & EUR/kW \\ \bottomrule
        \end{tabular}
    \end{small}
    \caption{ {\bf Technology cost uncertainty using
    optimistic and pessimistic assumptions from the Danish Energy
    Agency.\cite{DEA} }}
    \label{tab:costuncertainty}
\end{table}


\begin{figure}[H]
    \caption{ {\bf Cross-validation errors by output for
    varying sample sizes and polynomial orders of least-cost low-fidelity
    surrogate models. }}
    \label{fig:error}
\end{figure}